\documentclass[12pt]{article}
\usepackage{graphicx} % Required for inserting images
\usepackage{amsmath, amssymb, amsthm}
\usepackage{setspace, lipsum}
\usepackage[margin=2cm, headheight=15pt]{geometry}
\fontsize{24}{16}\selectfont
\setlength{\parindent}{0pt}

\usepackage{esdiff}
\usepackage{fixdif}


\renewcommand{\vec}[1]{\boldsymbol{#1}}
\DeclareMathOperator{\Span}{span}
\usepackage{xcolor}
\DeclareMathOperator{\rank}{rank}
\DeclareMathOperator{\kernel}{Ker}
\DeclareMathOperator{\image}{Im}
\DeclareMathOperator{\intd}{d}
\DeclareMathOperator{\trace}{tr}
\renewcommand{\qedsymbol}{$\blacksquare$}


\usepackage{fancyhdr}
\fancyhf{}
\rhead{\textit{Math 242 Written Assignment 5}}
\lhead{\textit{Written by Johnson 261105766}}
\chead{}
\cfoot\thepage
\pagestyle{fancy}


\begin{document}
\doublespacing

\section*{Question 1 (a)}

\begin{proof}
    To show that $\displaystyle{ \lim \left( \frac{n^2+n}{2n^2-3}
    \right) = \frac{1}{2}}$, we need to show that $\forall 
    \epsilon > 0$, $\exists N \in \mathbb{N} $, s.t 
    $\forall n \geq N$ : $ \displaystyle{\vert a_n -
    L \vert < \epsilon }$. 
    
    \vspace*{0.3cm}
    In this case, 
    $\displaystyle{(a_n) = \frac{n^2+2n}{2n^2-3}}$,
    $\displaystyle{L = \frac{1}{2}}$.

    
    \vspace*{0.3cm}
    $ \displaystyle{ \vert a_n - L \vert = 
    \bigg \vert \frac{n^2+n}{2n^2-3} - \frac{1}{2} \bigg \vert = 
    \bigg \vert \frac{2n+3}{4n^2-6} \bigg \vert }$.
    When $n$ is enfficently large, especially, say $n \geq 3$,
    then we 
    
    \vspace*{0.3cm}
    have $\displaystyle{ \bigg \vert \frac{2n+3}{4n^2-6} 
    \bigg \vert < \bigg\vert \frac{2n+3}{4n^2-9} \bigg\vert
    = \frac{2n+3}{4n^2-9}}$, because $2n+3 > 0$ and $
    4n^2 - 9 > 0$ in this case.

    \vspace*{0.3cm}
    We may observe that $\displaystyle{\frac{2n+3}{4n^2-9} = \frac{2n+3}
    {(2n+3)(2n-3)} = \frac{1}{2n-3}}$. This operation is totally
    doable because $2n-3>0$ and $2n+3>0$.

    \vspace*{0.3cm}
    Then, if we let $\displaystyle{\frac{1}{2n-3} < \epsilon}$,
    solve for $n$ we get $\displaystyle{ n >
    \frac{1}{\epsilon} + 3}$. Then, by our previous steps,
    we say that

    \[ \displaystyle{
        \vert a_n - L \vert < \frac{1}{2n-3} < \epsilon \hspace*{0.3cm}
        \text{for all } \hspace*{0.2cm} n > \frac{1}{\epsilon}+3
    }\]

    If we let $\displaystyle{N > \frac{1}{\epsilon}+3}$, then
    we conclude that $\forall n \geq N$, we have 
    $ \vert a_n - L \vert < \epsilon$, which means

    \[ \displaystyle{\lim \left( \frac{n^2+n}{2n^2-3} \right)
    = \frac{1}{2}} \]


\end{proof}

\newpage
\section*{Question 2}

\begin{proof}
By definition, if 
$\lim(x_n) = x$, it means 
$\exists N \in \mathbb{N}$, $\forall 
x \geq N$, $\vert x_n - x \vert < \epsilon$, for all
$\epsilon > 0$.
\begin{align*}
    &\Longrightarrow \vert x_n - x \vert < \epsilon \\
    &\Longrightarrow -\epsilon < x_n - x < \epsilon \\
    &\Longrightarrow -\epsilon+x < x_n < \epsilon+ x
\end{align*}
If we choose $\displaystyle{\epsilon = \frac{1}{2}x}$, then we have
$ \displaystyle{\frac{1}{2}x < x_n < \frac{3}{2}x}$, for all
$ x > N$ for some $N$. In this case, it is the same as saying
$\displaystyle{\frac{1}{2}x < x_n < 2x}$, because 
$x>0$. So we have proved the
inequility.




\end{proof}



\end{document}