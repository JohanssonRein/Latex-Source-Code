
\documentclass[12pt]{article}
\usepackage{graphicx} % Required for inserting images
\usepackage{amsmath, amssymb, amsthm}

\usepackage[margin=2cm, headheight=15pt]{geometry}
\fontsize{24}{16}\selectfont
\setlength{\parindent}{0pt}
\title{Linear Algebra}
\author{Review Material 1}
\date{2023 6}
\usepackage{esdiff}
\usepackage{fixdif}


\renewcommand{\vec}[1]{\boldsymbol{#1}}
\DeclareMathOperator{\Span}{span}
\usepackage{xcolor}
\DeclareMathOperator{\rank}{rank}
\DeclareMathOperator{\kernel}{Ker}
\DeclareMathOperator{\image}{Im}
\DeclareMathOperator{\intd}{d}
\DeclareMathOperator{\trace}{tr}



\usepackage{fancyhdr}
\fancyhf{}
\rhead{\textit{Math 235 Written Assignment 1}}
\lhead{\textit{Written by Johnson 261105766}}
\chead{}
\cfoot\thepage
\pagestyle{fancy}





\begin{document}



\part*{Content}

\vspace*{0.3cm}
The problems written in this document are: 1,2,4,5,6,8

\vspace*{4cm}

\[ \textbf{Problem 1} \cdots\cdots\cdots\cdots\cdots\cdots\textbf{page 2} \]

\[ \textbf{Problem 2} \cdots\cdots\cdots\cdots\cdots\cdots\textbf{page 4} \]

\[ \textbf{Problem 4} \cdots\cdots\cdots\cdots\cdots\cdots\textbf{page 7} \]

\[ \textbf{Problem 5} \cdots\cdots\cdots\cdots\cdots\cdots\textbf{page 9} \]

\[ \textbf{Problem 6} \cdots\cdots\cdots\cdots\cdots\cdots\textbf{page 10} \]

\[ \textbf{Problem 8} \cdots\cdots\cdots\cdots\cdots\cdots\textbf{page 12} \]

\newpage
\section*{Problem 1}

\begin{proof}

    In order to prove the theorem, we can construct an induction step as follows:
    
    \vspace*{0.3cm}
    \hspace*{1.2cm}
    \textbf{$\bullet$ The Base Step:}

    \vspace*{0.3cm}
    \hspace*{1.2cm}
    We firstly think of a set with only one element, $\mathcal{S}_1 = \lbrace{ a_1 \rbrace } $,
    
    \vspace*{0.3cm}
    \hspace*{1.2cm}
    $a_1 \in \mathbb{N} $
    In this case, $ \lvert \mathcal{S}_1 \rvert =1 $, so we can say that $a_1$ is the minimum element
    of $\mathcal{S}$
    
    \vspace*{0.3cm}
    \hspace*{1.2cm}
    Which we have shown that, the base case is true.

    \vspace*{0.3cm}
    \hspace*{1.2cm}
    \textbf{$\bullet$ The Inductive Step:}

    \vspace*{0.3cm}
    \hspace*{1.2cm}
    Now, for a picked $n$, we assume that the theorem is true, that is

    \[ \mathcal{S}_n = \lbrace a_1, a_2, \cdots, a_{n-1}, a_n \rbrace \hspace*{0.3cm} \text{has a minimum element, where}\]

    \[ a_1, a_2, \cdots, a_{n-1}, a_n \in \mathbb{N} \]

    \vspace*{0.3cm}
    \hspace*{1.2cm}
    Now we assume the minimum element in $\mathcal{S}_n$ is $a_k$, $1 \leq k \leq n$, so

    \[ \forall a_i \in \mathcal{S}_n, \hspace*{0.3cm} 1 \leq i \leq n, \hspace*{0.3cm} a_i \geq a_k \]

    \vspace*{0.3cm}
    \hspace*{1.2cm}
    Now, we would like to prove that the theorem is also true for $n+1$, so we construct 
    
    \vspace*{0.3cm}
    \hspace*{1.2cm}
    a new set $ \mathcal{S}_{n+1}$ with cardinality $n+1$ such that

    \[ \mathcal{S}_{n+1} = \lbrace a_1, a_2, \cdots, a_k, \cdots, a_n, a_{n+1} \rbrace \] 

    \vspace*{0.3cm}
    \hspace*{1.2cm}
    Now, we need to consider $a_{n+1}$

    \vspace*{0.3cm}
    \hspace*{2cm}
    $\blacktriangle$ \textit{IF $a_{n+1} > a_k$:}

    \vspace*{0.3cm}
    \hspace*{2cm}
    Then,by the proporty of $\mathcal{S}_n$ which we had already discussed, we know that

    \[ \forall a_i \in \mathcal{S}_{n+1}, \hspace*{0.3cm} 1 \leq i \leq n+1, \hspace*{0.3cm}, a_i \geq a_k \]

    \vspace*{0.3cm}
    \hspace*{2cm}
    So, $a_k$ is the minimum element in set $\mathcal{S}_{n+1}$

    \vspace*{0.3cm}
    \hspace*{2cm}
    So, there exists an minimum element in $\mathcal{S}_{n+1}$

    \vspace*{0.3cm}
    \hspace*{2cm}
    $\blacktriangle$ \textit{IF $a_{n+1} < a_k$:}

    \vspace*{0.3cm}
    \hspace*{2cm}
    Then,by the proporty of $\mathcal{S}_n$ which we had already discussed, we know that

    \[ \forall a_i \in \mathcal{S}_{n+1}, \hspace*{0.3cm} 1 \leq i \leq n+1, \hspace*{0.3cm}, a_i \geq a_{n+1} \]

    \vspace*{0.3cm}
    \hspace*{2cm}
    So, $a_k$ is the minimum element in set $\mathcal{S}_{n+1}$

    \vspace*{0.3cm}
    \hspace*{2cm}
    So, there exists an minimum element in $\mathcal{S}_{n+1}$

    \newpage
    \hspace*{2cm}
    $\blacktriangle$ \textit{IF $a_{n+1} = a_k$:}

    \vspace*{0.3cm}
    \hspace*{2cm}
    Then,by the proporty of $\mathcal{S}_n$ which we had already discussed, we know that

    \[ a_k = a_{n+1} \hspace*{0.2cm} \text{are the minumum elements of}  \hspace*{0.2cm}\mathcal{S}_{n+1} \]

    \vspace*{0.3cm}
    \hspace*{2cm}
    So, there exists an minimum element in $\mathcal{S}_{n+1} $

    \vspace*{0.3cm}
    \hspace*{1.2cm}
    $\bullet$ \textbf{Conclusion:}

    \vspace*{0.3cm}
    \hspace*{1.2cm}
    In conclusion, according to the base case and inductive case, it's true that

    \vspace*{0.3cm}
    \hspace*{1.2cm}
    Every subset of $\mathbb{N}$ has a minimum element.

\end{proof}

\newpage
\section*{Problem 2}

\begin{proof}
    We firstly noticed that, $\mathcal{C}_1 = 1, \mathcal{C}_2 = 2, \mathcal{C}_3 = 3 $

    \vspace*{0.3cm}
    \hspace*{1.2cm}
    So, it is clear that $\mathcal{C}_4 = 5, \mathcal{C}_5 = 8 $

    \vspace*{0.3cm}
    \hspace*{1.2cm}
    It is not hard to find the recursion formula for $\mathcal{C}_n$, which is:
    
    \[ \mathcal{C}_n = \mathcal{C}_{n-1} + \mathcal{C}_{n-2} ,\hspace*{0.3cm} n > 2 \]

    \vspace*{0.3cm}
    \hspace*{1.2cm}
    Then we need to prove by induction that, for each $n \geq 1$,

    \[ \mathcal{C}_{n} = \frac{1}{\sqrt{5}} \times \left[ \left( \frac{1+\sqrt{5}}{2} 
    \right)^{n+1} -  \hspace*{0.4cm}\left( \frac{1-\sqrt{5}}{2} \right)^{n+1} \right] \]
    
    \vspace*{0.3cm}
    \hspace*{1.2cm}
    $\bullet$ \textbf{The Base Case:}

    \vspace*{0.3cm}
    \hspace*{1.2cm}
    When $n=1$, we plug $n$ into the above formula, we get 
    

    \begin{align*}
        \mathcal{C}_{1} & = \frac{1}{\sqrt{5}} \times \left[ \left( \frac{1+\sqrt{5}}{2} 
    \right)^{2} -  \hspace*{0.4cm}\left( \frac{1-\sqrt{5}}{2} \right)^{2} \right] \\
        & = \frac{1}{4\sqrt{5}} \times \left[ \left( 1 + 2\sqrt{5}+5
        \right) -  \hspace*{0.4cm}\left( 1 - 2\sqrt{5}+5 \right) \right]\\
        & = \frac{1}{4\sqrt{5}} \times \left[ 4\sqrt{5} \right]\\
        & = 1 \\
        & = \mathcal{C}_1
    \end{align*}

    \vspace*{0.3cm}
    \hspace*{1.2cm}
    When $n=2$, we plug $n$ into the above formula, we get 
    

    \begin{align*}
        \mathcal{C}_{2} & = \frac{1}{\sqrt{5}} \times \left[ \left( \frac{1+\sqrt{5}}{2} 
    \right)^{3} -  \hspace*{0.4cm}\left( \frac{1-\sqrt{5}}{2} \right)^{3} \right] \\
        & = \frac{1}{8\sqrt{5}} \times \left[ \left( 8\sqrt{5}+16
        \right) -  \hspace*{0.4cm}\left( -8\sqrt{5}+16 \right) \right]\\
        & = \frac{1}{8\sqrt{5}} \times \left[ 16\sqrt{5} \right]\\
        & = 2 \\
        & = \mathcal{C}_2
    \end{align*}

    \vspace*{0.3cm}
    \hspace*{1.2cm}
    So, we have proved that the base case is correct.


    \vspace*{0.3cm}
    \hspace*{1.2cm}
    $\bullet$ \textbf{The Inductive Case:}

    \vspace*{0.3cm}
    \hspace*{1.2cm}
    For a picked $n$, $n \in \mathbb{N} , n \neq 0 $, we suppose the equation

    \[ \mathcal{C}_{n} = \frac{1}{\sqrt{5}} \times \left[ \left( \frac{1+\sqrt{5}}{2} 
    \right)^{n+1} -  \hspace*{0.4cm}\left( \frac{1-\sqrt{5}}{2} \right)^{n+1} \right] 
    \hspace*{0.3cm} \text{is true }\]

    \vspace*{0.3cm}
    \hspace*{1.2cm}
    Then, we need to prove that it is also true for $n+1$, i.e proving that

    \[ \mathcal{C}_{n+1} = \frac{1}{\sqrt{5}} \times \left[ \left( \frac{1+\sqrt{5}}{2} 
    \right)^{n+2} -  \hspace*{0.4cm}\left( \frac{1-\sqrt{5}}{2} \right)^{n+2} \right] 
    \hspace*{0.3cm} \text{is also true }\]

    \vspace*{0.3cm}
    \hspace*{1.2cm}
    From the recursion formula for $\mathcal{C}_n$ and the cases provided, we know that

    \[ \mathcal{C}_{n} = \mathcal{C}_{n-1} + \mathcal{C}_{n-2} \hspace*{0.3cm} \text{is true}\]

    \vspace*{0.3cm}
    \hspace*{1.2cm}
    If we denote $ \phi = \frac{\sqrt{5}+1}{2}$, then

    \[ \frac{1}{\sqrt{5}} \left[ (\phi)^{n+1} - \left( 1-\phi \right)^{n+1} \right] = 
    \frac{1}{\sqrt{5}} \left[ (\phi)^{n} - \left( 1-\phi \right)^{n} \right] + 
    \frac{1}{\sqrt{5}} \left[ (\phi)^{n-1} - \left(1 -\phi \right)^{n-1} \right] 
    \hspace*{0.3cm} \text{is true} \] 

    \vspace*{0.3cm}
    \hspace*{1.2cm}
    For $\mathcal{C}_{n+1} = \mathcal{C}_n + \mathcal{C}_{n-1} $, we need to prove that the formula still holds for

    \[ \frac{1}{\sqrt{5}} \left[ (\phi)^{n+2} - \left(1-\phi \right)^{n+2} \right] = 
    \frac{1}{\sqrt{5}} \left[ (\phi)^{n+1} - \left( 1-\phi \right)^{n+1} \right] + 
    \frac{1}{\sqrt{5}} \left[ (\phi)^{n} - \left( 1-\phi \right)^{n} \right] 
     \] 

    \vspace*{0.3cm}
    \hspace*{1.2cm}
    That is the same to prove

    \[   (\phi)^{n+2} - \left(1 -\phi \right)^{n+2}  = 
   (\phi)^{n+1} - \left( 1-\phi \right)^{n+1}  + 
     (\phi)^{n} - \left( 1-\phi \right)^{n}  \]

    \vspace*{0.3cm}
    \hspace*{1.2cm}
    Recall the properties of golden ratio $\phi = \frac{\sqrt{5}+1}{2}$, that I shall use in the following proof

    \[ \phi^2 = 1 + \phi \hspace*{2cm} \phi = 1 + \frac{1}{\phi} \]

    \begin{align*}
        L.H.S & =  (\phi)^{n+1} - \left( 1-\phi \right)^{n+1}  + 
        (\phi)^{n} - \left( 1-\phi \right)^{n} \\
         & = \phi^n \left( 1+\phi \right) - \left( 1-\phi \right)^n(1-\phi+1) \\
        & = \phi^n \left( 1+\phi \right) - \left( 1-\phi \right)^n(1+\phi-2\phi+1) \\
        & = \phi^n \left( \phi^2 \right) - \left( 1-\phi \right)^n(\phi^2-2\phi+1) \\
        & = \phi^{n+2} - \left( 1-\phi \right)^{n}\left(1 - \phi\right)^2 \\
        & = \phi^{n+2} - \left( 1-\phi \right)^{n+2}
    \end{align*}

    \vspace*{0.3cm}
    \hspace*{1.2cm}
    Which is exactly what we need to prove.

    \vspace*{0.3cm}
    \hspace*{1.2cm}
    In this case, we have proved that the formula is also true for $\mathcal{C}_{n+1}$

    \newpage
    \hspace*{1.2cm}
    $\bullet \textbf{Conclusion:}$

    \vspace*{0.3cm}
    \hspace*{1.2cm}
    According to the base case and the conductive case, we have proved that

    \[ \mathcal{C}_{n} = \frac{1}{\sqrt{5}} \times \left[ \left( \frac{1+\sqrt{5}}{2} 
    \right)^{n+1} -  \hspace*{0.4cm}\left( \frac{1-\sqrt{5}}{2} \right)^{n+1} \right] \]

    \[ \text{Is true for} \hspace*{0.2cm} n \geq 1, \hspace*{0.3cm} \mathcal{C}_{n+2} =
     \mathcal{C}_{n+1} + \mathcal{C}_{n}, \hspace*{0.2cm} \left( \mathcal{C}_{1} = 1
     , \mathcal{C}_{2} = 2 \right) \]
     
\end{proof}

\newpage
\section*{Problem 4}

\hspace*{0.6cm}To begin with, I would like to give a propsition that I will use to
solve this problem



\[\textit{Proposition:} \hspace*{0.3cm} 
1 + 2+ 3+ \cdots + n = \frac{(1+n)n}{2} \]




In general case, we may assume a sequence $\mathcal{A}_n$, 
such that $\mathcal{A}_n = \mathcal{X}n + \mathcal{Y}$,
$\mathcal{X},\mathcal{Y} \in \mathbb{R}$

\vspace{0.3cm}
So, we can give the formula below:

\[ \sum_{i=n}^{m} \mathcal{A}_i = \frac{\left( 
    \mathcal{A}_n+\mathcal{A}_m \right)(m-n+1)}{2} 
    \hspace*{0.3cm}\text{where} \hspace*{0.3cm}m > n \hspace*{0.3cm} m,n \in \mathbb{N}\]

Take a look at the elements of $\mathbb{N} \times \mathbb{N}$
using diagonals, we observe from the array that

\[ \begin{matrix}
    0&1&3&6&10&\cdots \\
    2&4&7&11& \cdots &\cdots \\
    5&8&12&\cdots&\cdots&\cdots \\
    9&13&\cdots&\cdots&\cdots&\cdots\\
    14&\cdots&\cdots&\cdots&\cdots&\cdots \\
    \vdots&\vdots&\vdots&\vdots&\vdots&\vdots
    
\end{matrix}\]

\vspace*{0.3cm}
In the first row, starting from the second element, we see that the elements are added by
1, 2, 3, 4, ... from the previous element; The second row is very similiar, which starts with 2 and
added by 2, 3, 4, ... accordingly.

\vspace*{0.3cm}
This is also true for columns, in the first column, the value of elements are added by 2,
 3, 4, 5, ... accordingly.




\[ \vdots \]

In this case, we may assume that if a number lies on the $i$th row and $j$th column, then
then the first value of $j$th column should be:

\[ \mathbb{N}_{(1,j) = }\underbrace{1+2+3+ \cdots + (j-1)}_\text{The first value of \textit{j}th column } 
= \frac{j(j-1)}{2}\]

\vspace*{0.3cm}
Therefore, by the obversation we had done before, if the number lies
on the $j$th column, then the second, third number in $j$th column 
should be 

\[ \mathbb{N}_{(2,j)} = \frac{j(j-1)}{2} + (j+1)\]

\[ \mathbb{N}_{(3,j)} = \frac{j(j-1)}{2} + (j+1)+(j+2)\]


So, in this case, if we want to reach the $i$th row, we need to sum up to $j+i-1$, which Is

\begin{align*}
    \mathbb{N}_{(i,j)} &= \frac{j(j-1)}{2} + (j+1) + (j+2) + \cdots + (j+i-1) \\
     &= \frac{j(j-1)}{2} + \frac{(2j+i)(i-1)}{2} \\
     & = \frac{i^2+j^2+2ij-i-3j}{2}
\end{align*}

\vspace*{0.3cm}
So we obtained a general formula for the elements of $\mathbb{N} \times
\mathbb{N}$ using diagonals.


In the array

\[ \begin{matrix}
    0&1&3&6&10&\cdots \\
    2&4&7&11& \cdots &\cdots \\
    5&8&12&\cdots&\cdots&\cdots \\
    9&13&\cdots&\cdots&\cdots&\cdots\\
    14&\cdots&\cdots&\cdots&\cdots&\cdots \\
    \vdots&\vdots&\vdots&\vdots&\vdots&\vdots
    
\end{matrix} \hspace*{3.8cm} \blacktriangle\]

If a number is located in the $i$th row and $j$th column, 
then the number $\mathbb{N}_{(i,j)}$ should be

\[ \mathbb{N}_{(i,j)} = \frac{i^2+j^2+2ij-i-3j}{2} \]

Now, we compare this formula to the rectangular array with coordinates:

\[ \begin{matrix}
    (0,0) & (0,1) & (0,2) & (0,3) & \cdots \\
    (1,0) & (1,1) & (1,2) & (1,3) & \cdots \\
    (2,0) & (2,1) & (2,2) & (2,3) & \cdots \\
    (3,0) & (3,1) & (3,2) & (3,3) & \cdots \\
    \vdots & \vdots &\vdots &\vdots &\vdots 
\end{matrix}  \hspace*{3cm} \blacklozenge\]

We found that the difference is that this array starts with
$0$th row and column, so if we want to apply the formula, we need to
add 1 to both the number of row and column.

\vspace*{0.3cm}
Say if a coordinate is (a,b) in the graph $\blacklozenge$, then the number related
to graph $\blacktriangle$ should be

\begin{align*}
     (a,b) & \Longrightarrow \frac{(a+1)^2+(b+1)^2+2(a+1)(b+1) -(a+1)-3(b+1)}{2} \\
     & = \frac{a^2+b^2+3a+b+2ab}{2}
\end{align*}

In that case, we can define a function $ f: \blacklozenge \rightarrow
\blacktriangle $, such that

\[ f(x,y) = \frac{x^2+y^2+3x+y+2xy}{2}\]

\vspace*{0.3cm}
Unsuprisingly, this is exactly the function we want, say that

\[ f: \mathbb{N}\times\mathbb{N} \rightarrow \mathbb{N} 
\hspace*{0.5cm} f(x,y) = \frac{x^2+y^2+3x+y+2xy}{2}
\hspace*{0.3cm} ( x,y \in \mathbb{N}) \]

\newpage
\section*{Problem 5}

\begin{proof}

To Show that $ \vert \mathcal{A}_1 \times \mathcal{B}_1 \vert
    = \vert \mathcal{A}_2 \times \mathcal{B}_2 \vert $

\vspace*{0.3cm}
\hspace*{1.2cm}
We may think of a function $ h = f \times g :\hspace*{0.3cm} \mathcal{A}_1 \times \mathcal{B}_1  \longrightarrow
\mathcal{A}_2 \times \mathcal{B}_2 \hspace*{0.3cm} \text{such that}$


\[ h(a,b) = ( f(a),g(b)) \hspace*{0.3cm} \text{where} 
\hspace*{0.3cm} a \in \mathcal{A}_1 , b \in \mathcal{B}_1\]

\vspace*{0.3cm}
\hspace*{1.2cm}
$\bullet$ \textbf{Firstly, to show the function is injective:}


\vspace*{0.3cm}
\hspace*{1.2cm}
Suppose $\exists x_1, x_2 \in \mathcal{A}_1$ and $y_1, y_2 
\in \mathcal{B}_1 $, such that $ h(x_1,y_1) = h(x_2,y_2)$

\vspace*{0.3cm}
\hspace*{1.2cm}
That is the same as saying $ (f(x_1),g(y_1) ) = (f(x_2),g(y_2))$

\vspace*{0.3cm}
\hspace*{1.2cm}
In order to be true, it satisifies $ f(x_1) = f(x_2) ,  g(y_1) = g(y_2) $


\vspace*{0.3cm}
\hspace*{1.2cm}
Since $ f: \mathcal{A}_1 \longrightarrow \mathcal{A}_2 \hspace*{0.3cm}\text{and}
\hspace*{0.3cm} g: \mathcal{B}_1 \longrightarrow \mathcal{B}_2 \hspace{0,3cm}\text{are both
bijective functions} $

\vspace*{0.3cm}
\hspace*{1.2cm}
So it says that $ x_1 =x_2$ and $y_1 = y_2$

\vspace*{0.3cm}
\hspace*{1.2cm}
So, $h$ is injective.

\vspace*{0.3cm}
\hspace*{1.2cm}
$\bullet$ \textbf{Secondly, to show the function is surjective:}

\vspace*{0.3cm}
\hspace*{1.2cm}
We may assume that $(x_0,y_0)$ is an arbitary element in $ \mathcal{A}_2
\times \mathcal{B}_2 $

\vspace*{0.3cm}
\hspace*{1.2cm}
If the function $h$ is bijective, then there will always exists $(x_*,y_*)$, such that

\begin{align*}
    h(x_*,y_*) &= (x_0,y_0) \\
    &= (f(x_*),g(y_*))
\end{align*}


\vspace*{0.3cm}
\hspace*{1.2cm}
Since $ f: \mathcal{A}_1 \longrightarrow \mathcal{A}_2 \hspace*{0.3cm}\text{and}
\hspace*{0.3cm} g: \mathcal{B}_1 \longrightarrow \mathcal{B}_2 \hspace{0,3cm}\text{are both
bijective functions} $

\vspace*{0.3cm}
\hspace*{1.2cm}
So, there will always have $x_*, y_*$ such that $x_* \mapsto x_0 , y_* \mapsto y_0 $

\vspace*{0.3cm}
\hspace*{1.2cm}
In this way, $h$ is surjective.

\vspace*{0.3cm}
\hspace*{1.2cm}
$\bullet$ \textbf{Conclusion:}

\vspace*{0.3cm}
\hspace*{1.2cm}
Since I have proved that $h$ is both injective and surjective, so $h$ is bijective

\vspace*{0.3cm}
\hspace*{1.2cm}
In other words, we have proved that $ \vert \mathcal{A}_1 \times \mathcal{B}_1 \vert
= \vert \mathcal{A}_2 \times \mathcal{B}_2 \vert $


\end{proof}

\newpage
\section*{Problem 6}

\begin{proof}
    In order to show $ \vert \mathbb{N} \vert = \vert \mathbb{Q} \vert $, we can prove the folowing
    two inequalities:

    \begin{align}
        \vert \mathbb{N} \vert & \leq \vert \mathbb{Q} \vert \\
        \vert \mathbb{Q} \vert & \leq \vert \mathbb{N} \vert 
    \end{align}

    \vspace*{0.3cm}
    \hspace*{1cm}
    $\bullet$ We may think of a function: $ f: \mathbb{N} \longrightarrow \mathbb{Q}$ such that
    

    \begin{equation*}
        f(x)=
            \begin{cases}
                \frac{1}{x} & \text{if } x \neq 0\\
                0 & \text{if } x = 0
            \end{cases}
        \end{equation*}

    \vspace*{0.3cm}
    \hspace*{1cm}
    Clearly, is this an injective function, because the $f$ is strictly increasing.
    
    \vspace*{0.3cm}
    \hspace*{1cm}
    So, if $f(x_1)=f(x_2)$, it must imply that $x_1 = x_2$.

    \vspace*{0.3cm}
    \hspace*{1cm}
    So by the property of sets and functions, we say that 
    $ \vert \mathbb{N} \vert \leq \vert \mathbb{Q} \vert$ is true.

    \vspace*{0.3cm}
    \hspace*{1cm}
    $\bullet$ Now, think of a function $g$ that takes a rational number $p$ as input, say

    \[ p = \frac{a}{b} \hspace*{0.3cm} \text{where} \hspace*{0.3cm} a,b \in \mathbb{Z}, \hspace*{0.3cm} b \neq 0 \]

    \vspace*{0.3cm}
    \hspace*{1cm}
    And let our output to be a 2-D array with coordinates $a,b$. Then define:

    \[ g: \mathbb{Q} \longrightarrow \mathbb{Z} \times \mathbb{Z},
    \hspace*{0.4cm} g \left( \frac{a}{b} \right) = (a,b) \]

    \vspace*{0.3cm}
    \hspace*{1cm}
    By the property of rational numbers, if $p_1 \neq p_2$, then

    \[ \left( \frac{a_1}{b_1} \right) \neq \left( \frac{a_2}{b_2} \right) \]

    \vspace*{0.3cm}
    \hspace*{1cm}
    So, $a_1 = a_2$ and $b_1 = b_2$ cannot be true at the same time, that is 

    \[ (a_1,b_1) \neq (a_2,b_2) \]

    \vspace*{0.3cm}
    \hspace*{1cm}
    So, the function $g$ is also injective. That is, $ \vert \mathbb{Q}\vert \leq \vert \mathbb{Z} \times \mathbb{Z} \vert $

    \vspace*{1cm} 
    \hspace*{1cm}
    Now let's show that $\vert \mathbb{Z} \times \mathbb{Z} \vert 
    = \vert \mathbb{N} \vert $
    by constructing another function $ \mathbb{Z} \times \mathbb{Z} 
    \mapsto \mathbb{N} $

    \vspace*{1cm}
    \hspace*{1cm}
    Think of a 2-D coordinates with all integer pairs $(a,b)$,
    take a look at a very small range.

    \[ \begin{matrix}
        (-1,1) & (0,1) & (1,1) \\
        (-1,0) & (0,0) & (1,0) \\
        (-1,-1) & (0,-1) & (1,-1)
    \end{matrix} \]

    \vspace*{0.3cm}
    \hspace*{1cm}
    If an operation starts from $(0,0)$ and takes vertical/horizontal moves
    as follows:

    \[ \begin{matrix}
        & & & &\cdots &\longleftarrow & (2,2) \\
        & & & & & & \uparrow \\
        (-1,1) &\longleftarrow &(0,1) &\longleftarrow & (1,1)& &(2,1) \\
        \downarrow& & & &\uparrow& &\uparrow \\
        (-1,0) & & (0,0) &\longrightarrow & (1,0) & & (2,0)\\
        \downarrow& & & & & &\uparrow\\
        (-1,-1) &\longrightarrow &(0,-1) & \longrightarrow&(1,-1) & \longrightarrow&(2,-1)
    \end{matrix} \]

    \vspace*{0.3cm}
    \hspace*{1cm}
    Now we consider a sequence $\mathcal{C}(a,b) = x$, $x \in \mathbb{N}$,
     $a,b \in \mathbb{Z}$ and let $x$ to be the total times 
     
    \vspace*{0.3cm}
    \hspace*{1cm}
    of operations that starts from
     $(0,0)$ and ends with $(a,b)$

    \vspace*{0.3cm}
    \hspace*{1cm}
    For example, $\mathcal{C}(0,0) = 0$, $\mathcal{C}(1,1) = 2$
    , $\mathcal{C}(2,-1) = 9$

    
    \vspace*{0.3cm}
    \hspace*{1cm}
    Then we can define our function to be:

    \[ \mathcal{C}: \mathbb{Z}\times \mathbb{Z} \longrightarrow
    \mathbb{N}, \hspace*{0.3cm} \mathcal{C}(a,b) = x \]

    \vspace*{0.3cm}
    \hspace*{1cm}
    Since we had proved in class that $\vert \mathbb{Z} \vert 
    = \vert \mathbb{N} \vert $ by consturcting a bijective function

    \[ h: \mathbb{N} \mapsto \mathbb{Z}, \hspace*{0.3cm} \text{such that} \]
    
    
    \begin{equation*}
        h(x)=
            \begin{cases}
                2x & \text{if } x \geq 0\\
                -(2x+1) & \text{if } x < 0
            \end{cases}
        \end{equation*}
    
    
    \vspace*{1cm}
    \hspace*{1cm}
    So, $\mathcal{C}(a,b)$ is bijective, because each different input
    maps to a unique output.

    \vspace*{0.3cm}
    \hspace*{1cm}
    And for all $x\in \mathbb{N}$, there always exists an input that maps to it.

    \vspace*{0.3cm}
    \hspace*{1cm}
    So, it is true that $\vert \mathbb{Z}\vert \times \vert \mathbb{Z} \vert = \vert \mathbb{N} \vert$



    \vspace*{1cm}
    \hspace*{1cm}
    So, it applies that $\vert \mathbb{Q}\vert \leq \vert \mathbb{Z} \times \mathbb{Z} \vert
    = \vert \mathbb{N} \vert $
    
    \vspace*{1cm}
    \hspace*{1cm}
    $\bullet$ By now, we proved that there exists an injection between two directions. 

    \vspace*{0.3cm}
    \hspace*{1cm}
    So both the inequalities $(1), (2)$ are correct, that is $\vert \mathbb{N} \vert =
    \vert \mathbb{Q} \vert $.

\end{proof}

\newpage
\section*{Problem 8}

\subsection*{First proof: $\vert \mathcal{A} \vert \leq \vert 2^{\mathcal{A}} \vert$}

\begin{proof}
    Suppose $\vert \mathcal{A} \vert = n$, and $a_1, a_2, \cdots, a_n$ are the elements in $\mathcal{A}$, 

    \vspace*{0.3cm}
    \hspace*{1cm}
    We can construct a function $f$ with $m$ variables, which takes $m$ elements from
    $\mathcal{A}$, 
    
    \vspace*{0.3cm}
    \hspace*{1cm}
    $(0 \leq m \leq n)$ as input, and a set containing these elements as output, 

    \[ \text{Suppose} \hspace*{0.3cm} \vec a = \begin{pmatrix}
        a_1 \\
        a_2 \\
        \vdots \\
        a_m
    \end{pmatrix} , \hspace*{0.2cm} \text{then define} \hspace*{0.2cm} 
    f^{m}(\vec{a}) = \lbrace a_1, a_2, \cdots, a_m \rbrace \]

    \vspace*{0.3cm}
    \hspace*{1cm}
    We need to show that for any $m$, $f^m$ is injective.

    \vspace*{0.3cm}
    \hspace*{1cm}
    Suppose $\vec{a} = \vec{b}$, then we know that the vectors $\vec{a},\vec{b}$ where

    \[ \vec a = \begin{pmatrix}
        a_1 \\
        a_2 \\
        \vdots \\
        a_m
    \end{pmatrix} , \vec b = \begin{pmatrix}
        b_1 \\
        b_2 \\
        \vdots \\
        b_m
    \end{pmatrix} \hspace*{0.2cm} \text{must be equal}\]

    \vspace*{0.3cm}
    \hspace*{1cm}
    So the $i$th coordinates $( 1 \leq i \leq m)$ of $\vec a,\vec b$ are equal

    \vspace*{0.3cm}
    \hspace*{1cm}
    So, $ a_1 = b_1; a_2 = b_2; \ldots; a_m = b_m $

    \vspace*{0.3cm}
    \hspace*{1cm}
    Then, it's also true that $ f^m(\vec a) = f^m(\vec b)$, so the function is injective. In this case

    \[ \vert \mathcal{A} \vert \leq \vert 2^{\mathcal{A}} \vert \]

\end{proof}

\section*{Second Proof:}

\begin{proof}
    
    Let's think of a set $\mathcal{A}$ with specific elements, say $\mathcal{A}
    = \lbrace 1, 2, 3, 4, 5 \rbrace$, then

    \[ \mathcal{A}_1 = \lbrace 1,2,3\rbrace ; \mathcal{A}_2 = \lbrace 4,5\rbrace; 
    \mathcal{A}_3 = \lbrace 1,2,4,5 \rbrace; \mathcal{A}_4 = \lbrace 4,5 \rbrace \]
    
    \vspace*{0.3cm}
    \hspace*{1cm}
    are subsets of $\mathcal{A}$, but not all of them are listed.
    Now, we can define a function $ g : a \mapsto \mathcal{A}_a $

    \vspace*{0.3cm}
    \hspace*{1cm}
    For example: $ g(1) = \mathcal{A}_1 = \lbrace 1, 2, 3 \rbrace $, $ g(3) = 
    \lbrace 1, 2, 4 ,5 \rbrace $ and so on.

    \vspace*{0.3cm}
    \hspace*{1cm}
    We can define a subset of $\mathcal{A}$ to be $\mathcal{U}$, such that $ \mathcal{U} :
    \lbrace x \in \mathcal{A} : x \notin g(x) \rbrace  $

    \vspace*{0.3cm}
    \hspace*{1cm}
    So, let's take a look at the previous example, $ 1 \notin \mathcal{U} $ because
    $ 1 \notin g(1) $. Then We say 1 is $\textit{non-bounded}$

    \vspace*{0.3cm}
    \hspace*{1cm}
    $ 2 \in \mathcal{U} $ because $ 2 \notin g(2) = \lbrace 4, 5 \rbrace $.
     Then we say 2 is $\textit{bounded}$

    \vspace*{0.3cm}
    \hspace*{1cm}
    Now, let's construct a bijection between $ \mathcal{A} $ and $2^{\mathcal{A}}$

    \newpage
    \hspace*{1cm}
    Suppose $ \lambda $
    is $bounded$, then $\lambda \in \mathcal{U}$, that contradicts to the definition
    
    \vspace*{0.3cm}
    \hspace*{1cm}
    Suppose  $\lambda$ is $\textit{non-bounded}$, then it means that $\lambda$ should be in $\mathcal{U}$
    instead
    
    
    \vspace*{0.3cm}
    \hspace*{1cm}
    So the above give us contradiction, thus the function cannot be bijective, ie 

    \[ \vert \mathcal{A} \vert < \vert 2^{\mathcal{A}} \vert \]
    



\end{proof}







\end{document}