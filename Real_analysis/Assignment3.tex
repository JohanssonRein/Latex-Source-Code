\documentclass[12pt]{article}
\usepackage{graphicx} % Required for inserting images
\usepackage{amsmath, amssymb, amsthm}
\usepackage{setspace, lipsum}
\usepackage[margin=2cm, headheight=15pt]{geometry}
\fontsize{24}{16}\selectfont
\setlength{\parindent}{0pt}

\usepackage{esdiff}
\usepackage{fixdif}
\renewcommand{\qedsymbol}{$\blacksquare$}

\renewcommand{\vec}[1]{\boldsymbol{#1}}
\DeclareMathOperator{\Span}{span}
\usepackage{xcolor}
\DeclareMathOperator{\rank}{rank}
\DeclareMathOperator{\kernel}{Ker}
\DeclareMathOperator{\image}{Im}
\DeclareMathOperator{\intd}{d}
\DeclareMathOperator{\trace}{tr}


\usepackage{fancyhdr}
\fancyhf{}
\rhead{\textit{Math 242 Written Assignment 3}}
\lhead{\textit{Written by Johnson 261105766}}
\chead{}
\cfoot\thepage
\pagestyle{fancy}


\begin{document}
\doublespacing

\section*{Problem 3(a)}

\begin{proof}
    To prove $A$ is bounded and has supremum and infimum,
    show that:
    
    $\bullet \hspace*{0.3cm}\textbf{Bounded from above
    :}$

    Noticed that $A := \displaystyle{\Big\lbrace 2+
    \frac{(-1)^n}{n} : n
    \in \mathbb{N} \Big\rbrace} $, for $x \in A$, we have
    $ x = 2 + \displaystyle\frac{(-1)^n}{n} $.

    \vspace*{0.3cm}
    Since we know that $ \displaystyle\frac{(-1)^n}{n}
    < \displaystyle\frac{1}{n}$, so it is clear that 
    $ 2 + \displaystyle \frac{(-1)^n}{n} <
    2 + \displaystyle\frac{1}{n} \leq 2+1 = 3$.

    \vspace*{0.3cm}
    Then we can say that $A$ is bounded from above because
    $\forall x \in A, x \leq 3$.
    
    \vspace*{0.3cm}
    $\bullet \hspace*{0.3cm} \textbf{Bounded from below :}$
    
    Similarily, we know that $\displaystyle\frac{(-1)^n}{n}
    \geq - \displaystyle \frac{1}{n}$, so it is clear that
    $ 2+\displaystyle\frac{(-1)^n}{n} \geq 2-\displaystyle
    \frac{1}{n} \geq 2-1 = 1$.

    \vspace*{0.3cm}
    Then we say that $A$ is bounded from below because 
    $\forall x \in A, x \geq 1$.

    \vspace*{0.3cm}
    Since $1 \in A$, and $\forall x \in A, x \geq 1$, We
    may conclude that $\inf(A) = 1$.

    \vspace*{0.3cm}
    $\bullet \hspace*{0.3cm} \textbf{Supremum : }$
    
    Assume that $\sup(A) = \mu$, this means that
    $\forall x \in A$, $\mu \geq x$.

    \vspace*{0.3cm}
    By the justifications before, we know that 
    $2+\displaystyle \frac{(-1)^n}{n} < 2+\displaystyle
    \frac{1}{n}$, when $n$ is even,  $(-1)^n \geq 0$
    
    \vspace*{0.3cm}
    Since 2 is the minimum even  number in $\mathbb{N}$,
    it applies that $2+\displaystyle\frac{(-1)^n}{n} 
    = 2+\displaystyle\frac{1}{n}$, for even numbers $n$.

    \vspace*{0.3cm}
    Similarily, $2+\displaystyle\frac{(-1)^n}{n} = 2-
    \displaystyle \frac{1}{n}$, for odd numbers $n$.

    \vspace*{0.3cm}
    In this case, we may conclude that $\sup(A)\geq2$ because
    for each $n$ to be even the statement is always true, 
    and if $n$ is odd, the result is always less than 2.

    \vspace*{0.3cm}
    Also, we know that $f(n) = \displaystyle \frac{1}{n}$ is
    strictlt decreasing, so for even numbers $n$,
    $\displaystyle \frac{1}{2} \geq \displaystyle 
    \frac{1}{n}$.

    \vspace*{0.3cm}
    So, $2 + \displaystyle\frac{(-1)^n}{n} \leq
    \displaystyle 2+\frac{1}{2} = \displaystyle 
    \frac{5}{2}$. By definition, we then conclude that
    $\sup(A) = \displaystyle\frac{5}{2}$

    \vspace*{0.3cm}
    $\bullet \hspace*{0.3cm} \textbf{Infimum :}$

    As shown in the bounded from below part, we already know that
    $\inf(A) = 1$.


    \vspace*{0.2cm}
    Finally, we have proved that $A$ is bounded and we
    know that $\sup(A) = \displaystyle \frac{5}{2}$
    and $\inf(A) = 1$.

\end{proof}

\section*{Problem 3(b)}
\begin{proof}

    To prove $B$ is bounded and has supremum and infimum,
    show that:

    $\bullet \hspace*{0.3cm} \textbf{Bounded from 
    above:} $

    To show that $B := \displaystyle{\Big\lbrace (-1)^n+
    \frac{1}{n} : n
    \in \mathbb{N} \Big\rbrace} $, for $x \in B$, we have
    $ x = (-1)^n + \displaystyle\frac{1}{n} $.

    \vspace*{0.3cm}
    Since we know that $ \displaystyle\frac{1}{n}+(-1)^n <
    \displaystyle\frac{1}{n}+1 \leq 2$, so we can say that $B$ is bounded from above,
     because $ \forall x \in B, x \leq 2$. And obviously 2 is an upper bound.

    $\bullet \hspace*{0.3cm} \textbf{Bounded from below :}$
    
    \vspace*{0.3cm}
    Since we know that $ (-1)^n \geq -1$,and $\displaystyle\frac{1}{n} > 0$, so
    $\displaystyle\frac{1}{n}+(-1)^n \geq -1$, then we say that $B$ has a lower bound,
    because $\forall x \in B, x \geq -1$, thus $B$ is bounded from below.

    $\bullet \hspace*{0.3cm} \textbf{Supremum :}$
    
    Since when $n$ is even, $2 > (-1)^n + \displaystyle\frac{1}{n} > 1$, when $n$ is odd,
    $-1 < (-1)^n + \displaystyle \frac{1}{n} < 0.$ It implies that
    $2 > \sup(B) > 1$. Again the function $f(x) = \displaystyle\frac{1}{x}$ is strictly
    decreasing, so $ \displaystyle \frac{1}{2} \geq \displaystyle\frac{1}{n}$ for all
     even $n$.

    \vspace*{0.3cm}
    In this case, $ \displaystyle \frac{1}{n} + (-1)^n \leq \displaystyle\frac{1}{2}+1
    = \displaystyle \frac{3}{2}.$ So $\sup(B) = \displaystyle \frac{3}{2}$.

    $\bullet \hspace*{0.3cm} \textbf{Infimum}$

    Similarily, we know that $\displaystyle \frac{1}{n} + (-1)^n > -1$, when $n$ is odd,
    since $\displaystyle \frac{1}{n} \leq 1$, so $\displaystyle \frac{1}{n} + (-1)^n \leq 0$

    \vspace*{0.3cm}
    Then, we may say that $\sup(B) \leq 0$. Since $f(x) = \displaystyle \frac{1}{x}$ is strictly
    decreasing, when $x$ gets larger, $f(x)$ will approaching 0.

    \vspace*{0.3cm}
    So in this case, $\displaystyle \frac{1}{n}+(-1)^n \geq -1$, for all $ n \in \mathbb{N}$.
    Then we conclude that $\inf(B) = -1$.

    \vspace*{0.3cm}
    Finally, we have proved that $B$ is bounded and we
    know that $\sup(A) = \displaystyle \frac{3}{2}$
    and $\inf(A) = -1$.

\end{proof}

\newpage
\section*{Problem 6}

\begin{proof}
    We know that $S$ is bounded from above, meaning that $\exists \lambda \in \mathbb{R} :
    \forall s \in S, s \leq \lambda$I in this case, $\lambda$ is an upper bound of $S$. If $ k \geq 0$, 
    then it is still true that $ ks \leq k\lambda$.

    \vspace*{0.3cm}
    By definition, we know that $ ks \in kS$, so $\forall t \in kS  :   t \leq k\lambda$.
    Then we say $kS$ is bounded from above. As $k\lambda$ is an upper bound of $kS$.

    \vspace*{0.3cm}
    Then we assume that $\sup(S) = \mu$, meaning that $\mu$ is the smallest upper bound of $S$,
    i.e, for all upperbounds $\lambda, \lambda \geq \mu$.
    So $\forall s \in S : s \leq \mu \leq \lambda$.

    \vspace*{0.3cm}
    Similarily, if $k \geq 0$, then $ks \leq k\mu \leq k\lambda$. By definition, 
    $ks \in kS $, so $k\mu$ is the supremum of $kS$.

    \vspace*{0.3cm}
    Then, $\sup(S) = \mu$ and $\sup(kS) = k\mu$. So we conclude that
    $\sup(kS) = k\sup(S)$.





    


\end{proof}



\end{document}
