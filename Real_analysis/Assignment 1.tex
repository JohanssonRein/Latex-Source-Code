\documentclass[12pt]{article}
\usepackage{graphicx} % Required for inserting images
\usepackage{amsmath, amssymb, amsthm}

\usepackage[margin=2cm, headheight=15pt]{geometry}
\fontsize{24}{16}\selectfont
\setlength{\parindent}{0pt}

\usepackage{esdiff}
\usepackage{fixdif}

\renewcommand{\vec}[1]{\boldsymbol{#1}}
\DeclareMathOperator{\Span}{span}
\usepackage{xcolor}
\DeclareMathOperator{\rank}{rank}
\DeclareMathOperator{\kernel}{Ker}
\DeclareMathOperator{\image}{Im}
\DeclareMathOperator{\intd}{d}
\DeclareMathOperator{\trace}{tr}

\usepackage{fancyhdr}
\fancyhf{}
\rhead{\textit{Math 242 Written Assignment 1}}
\lhead{\textit{Written by Johnson 261105766}}
\chead{}
\cfoot\thepage
\pagestyle{fancy}
\begin{document}


\section*{Problem 1}

\subsection*{1.1 part (a)}

\begin{proof}
    
    \vspace*{0.3cm}
    In order to prove $ \forall x,y \in \mathbf{R}: 
    y^3 - x^3 = (y-x)(x^2 + xy + y^2)$,

    \vspace*{0.5cm}
    \hspace*{1.2cm}
    We can expand the $R.H.S$ of the equation,that is
    \begin{align*}
        (y-x)(x^2+xy+y^2) &= yx^2+xy^2+y^3-x^3-x^2y-xy^2 \\
        \bullet & = y^3-x^3+(yx^2-yx^2)+(xy^2-xy^2) \\
        & = y^3-x^3
    \end{align*}

    \hspace*{1.2cm}Those two parts got cancelled in step $\bullet$

    \vspace*{0.3cm}
    \hspace*{1.2cm}
    So for the right hand side, we get $R.H.S = y^3-x^3$
    
    \vspace*{0.3cm}
    \hspace*{1.2cm}
    Compared to the origional equation, we found that

    \[(y-x)(x^2+xy+y^2) = y^3-x^3\]

    \vspace*{0.3cm}
    \hspace*{1.2cm}
    That is, $L.H.S = R.H.S$

    \vspace*{0.3cm}
    \hspace*{1.2cm}
    So, we have proved that $y^3 - x^3 = (y-x)(x^2 + xy + y^2)$.
        
    

\end{proof}

\subsection*{1.2 part (b)}

\vspace{0.3cm}

\begin{proof}
    
    In order to prove $ \forall x, y \in \mathbf{R} : x^2 +xy + y^2 \geq 0$,

    \vspace*{0.3cm}
    \hspace*{1.2cm}
    Using direct proof, we firstly noticed that:

    \[ (x+y)^2 \geq 0 : \forall x,y \in \mathbf{R}\]

    \vspace*{0.3cm}
    \hspace*{1.2cm}
    By expanding the $L.H.S$ of the above,we get :

    \[ x^2 + 2xy + y^2 \geq 0 \]

    \vspace*{0.3cm}
    \hspace*{1.2cm}
    Then, we substract both $L.H.S$ and $R.H.S$ by $xy$, and still holds for:

    \[ x^2 + xy + y^2 \geq -xy \hspace*{2cm} \blacktriangle \]

    \vspace*{0.3cm}
    \hspace*{1.2cm}
    Thus, there are three cases for this:

    \vspace*{0.5cm}
    \hspace*{2.5cm}
    $\bullet$ Case 1: If $x,y$ has the same sign, (i.e $xy > 0$ ) then

    \vspace*{0.3cm}
    \hspace*{2.9cm}
    From the $L.H.S$ of $\blacktriangle$, $x^2 > 0;  y^2 > 0;  xy>0$

    \vspace*{0.3cm}
    \hspace*{2.9cm}
    So it is obvious that $x^2 + xy +y^2 > 0$

    \newpage
    \hspace*{2.5cm}
    $\bullet$ Case 2: If $x,y$ has the different sign, (i.e $xy < 0$ ) then

    
    \vspace*{0.3cm}
    \hspace*{2.9cm}
    It is obvious that $-xy > 0$ instead

    \vspace*{0.3cm}
    \hspace*{2.9cm}
    Then from the $R.H.S$ of $\blacktriangle$, we see that $R.H.S > 0$

    \vspace*{0.3cm}
    \hspace*{2.9cm}
    Since the inequality says that $L.H.S > R.H.S$

    \vspace*{0.3cm}
    \hspace*{2.9cm}
    So it demonstrates that $x^2 + xy + y^2 > -xy > 0$

    \vspace*{0.3cm}
    \hspace*{2.9cm}
    Which is $x^2 + xy + y^2 > 0$


    \vspace*{0.5cm}
    \hspace*{2.5cm}
    $\bullet$ Case 3: If $xy = 0$ then

    \vspace*{0.3cm}
    \hspace*{2.9cm}
    For cases like $x=0$ $\mathbf{or}$ $y=0$

    \vspace*{0.3cm}
    \hspace*{2.9cm}
    It is clear that from $\blacktriangle$, $L.H.S >0$ and $R.H.S =0$, the inequality holds

    \vspace*{0.3cm}
    \hspace*{2.9cm}
    For a special case, where $x=0$ $\mathbf{and}$ $y=0$


    \vspace*{0.3cm}
    \hspace*{2.9cm}
    In this case only, from $\blacktriangle$ we can apply that $L.H.S = R.H.S = 0$


    \vspace*{0.5cm}
    \hspace*{1.2cm}
    In conclusion, we can say that $\forall x,y \in \mathbf{R}, x^2 + xy + y^2 \geq 0$

    \vspace*{0.5cm}
    \hspace*{1.2cm}
    $ x^2 + xy + y^2 = 0 $ \textit{if and only if} $x = 0, y = 0$


\end{proof}


\subsection*{1.3 part (c)}

\vspace*{0.3cm}
\begin{proof}

    In order to prove $ f : \mathbf{R} \longrightarrow \mathbf{R},
     x \mapsto x^3 $ is strictly increasing, we need to show

    \vspace*{0.3cm}
    \hspace*{1.2cm}
    $ \forall x,y \in \mathbf{R}$, if $x>y$, then $f(x) > f(y)$

    \vspace*{0.3cm}
    \hspace*{1.2cm}
    That is, prove $x^3 > y^3$ if $x > y$

    \vspace*{0.3cm}
    \hspace*{1.2cm}
    By direct proof, if $x>y$, then we can conclude that $x-y > 0$

    \vspace*{0.3cm}
    \hspace*{1.2cm}
    As shown in part (b), $\forall x,y \in \mathbf{R}, x^2 + xy + y^2 \geq 0 $

    \vspace*{0.3cm}
    \hspace*{1.2cm}
    Since $x \neq y$, which means $x,y$ cannot be all zero

    \vspace*{0.3cm}
    \hspace*{1.2cm}
    So the statement $ x^2 + xy + y^2 >0$ is true for this case

    \vspace*{0.3cm}
    \hspace*{1.2cm}
    So $ (x^2 + xy + y^2)(x-y) > 0$ is also true

    \vspace*{0.3cm}
    \hspace*{1.2cm}
    By expanding the formula above, we get $ x^3 - yx^2+x^2y-xy^2+xy^2-y^3 > 0 $

    \vspace*{0.3cm}
    \hspace*{1.2cm}
    That is, $ x^3 - y^3 > 0 $, which is exactly what we need to prove

    \vspace*{0.3cm}
    \hspace*{1.2cm}
    So, $ f : \mathbf{R} \longrightarrow \mathbf{R},
     x \mapsto x^3 $ is strictly increasing.

\end{proof}

\newpage
\section*{Problem 2 and Problem 3 are omitted}




\newpage
\section*{Problem 4}

\subsection*{4.1 part (a)}

\begin{proof}
    In order to show that $\sqrt{6}$ is irrational,

    \vspace*{0.3cm}
    \hspace*{1.2cm}
    Using proof by contradiction, assume that $\sqrt{6}$ is rational, so it can be written as

    
    \[ \sqrt{6} = \frac{a}{b}; \hspace*{0.3cm}  a,b \in \mathbb{N};
    \hspace*{0.3cm}   b \neq 0;  \hspace*{0.3cm}  \gcd(a,b) = 1 \]
    \begin{align*}
        \Longrightarrow 6 & = \frac{a^2}{b^2} \\
        \Longrightarrow 6b^2 & = a^2
    \end{align*}

    \hspace*{1.2cm}
    Since $6b^2$ must be an even number, it applies that $a^2$ is also an even number

    
    \vspace*{0.3cm}
    \hspace*{1.2cm}
    So, $a$ is an even number, also $a$ is divisible by 6, meaning that

    \begin{align*}
        \exists c \in \mathbb{N} : a  & = 6c \\
        \Longrightarrow 6b^2  & = 36c^2 \\
        \Longrightarrow b^2 & = 6c^2
    \end{align*}

    \hspace*{1.2cm}
    Since $6c^2$ must be an even number, it applies that $b^2$ is also an even number

    
    \vspace*{0.3cm}
    \hspace*{1.2cm}
    So, $b$ is an even number

    \vspace*{0.3cm}
    \hspace*{1.2cm}
    If both $a,b$ are even, meaning that $\gcd(a,b) \geq 2 $, thus
    $ \gcd(a,b) \neq 1$

    \vspace*{0.3cm}
    \hspace*{1.2cm}
    Which leads to a contradiction. So $\sqrt{6}$ must be irrational.

    
\end{proof}

\subsection*{4.2 part (b)}

\begin{proof}
    In order to prove that $\sqrt 2 + \sqrt 3 $ is irrational,

    \vspace*{0.3cm}
    \hspace*{1.2cm}
    Using proof by contradiction, assume that both$\sqrt 2, \sqrt 3 $ are rational

    \vspace*{0.3cm}
    \hspace*{1.2cm}
    So $\sqrt{2} + \sqrt 3$  is rational

    \vspace*{0.3cm}
    \hspace*{1.2cm}
    By taking the square, $ (\sqrt 2 + \sqrt 3)^2$ is also rational

    \vspace*{0.3cm}
    \hspace*{1.2cm}
    By expanding the square, $ 2 + 3 + \sqrt{2 \times 3} $ is also rational

    \vspace*{0.3cm}
    \hspace*{1.2cm}
    Which is, $ 5 + \sqrt 6 $ is rational

    \vspace*{0.3cm}
    \hspace*{1.2cm}
    Since 5 is rational, and $\sqrt 6$ is irrational (shown in part a)

    \vspace*{0.3cm}
    \hspace*{1.2cm}
    Then $5+\sqrt 6$ can not be rational, that is $5+\sqrt 6$ is irrational. 

    \vspace*{0.3cm}
    \hspace*{1.2cm}
    That leads to a contradiction, so $\sqrt 2 + \sqrt 3$ is irrational.

\end{proof}

\newpage
\section*{Problem 5}

\subsection*{5.1 part (a)}

\begin{proof}
    In order to prove 
    
    \[ \underbrace{\sqrt{2+\sqrt{2+\sqrt{2+\cdots + \sqrt{2}}}} }
    _\text{n nested square roots} = 2\cos
    \left(  \frac{\pi}{2^{n+1}}  \right) \]
    
    \vspace*{0.3cm}
    \hspace*{1.2cm}
    By induction, we know that the range of n is $ n \in \mathbf{N}$

    \vspace*{0.3cm}
    
    $\bullet$ \textbf{The Base Case :}

    \vspace*{0.3cm}
    \hspace*{1.2cm}
    $\blacktriangle$  When $n=1$ 
    
    \vspace*{0.3cm}
    \hspace*{1.2cm}
    From the $L.H.S$ of the equation $\clubsuit$, we know $L.H.S = \sqrt 2$

    \vspace*{0.3cm}
    \hspace*{1.2cm}
    Also, $R.H.S = 2\cos \left( \frac{\pi}{4} \right) = 2\times \frac{\sqrt 2}{2} = \sqrt 2 $

    \vspace*{0.3cm}
    \hspace*{1.2cm}
    So, $L.H.S = R.H.S$ for base case, which is true.

    \vspace*{0.3cm}
    
    $\bullet$ \textbf{The Inductive Case :}

    \vspace*{0.3cm}
    \hspace*{1.2cm}
    $\blacktriangle$ Suppose $\clubsuit$ is true for one $n \in \mathbb{N}$, that is

    \[ \underbrace{\sqrt{2+\sqrt{2+\sqrt{2+\cdots + \sqrt{2}}}} }
    _\text{n nested square roots} = 2\cos
    \left(  \frac{\pi}{2^{n+1}}  \right) \]

    \vspace*{0.3cm}
    \hspace*{1.2cm}
    $\blacktriangle$ Then we need to prove it also holds for $n+1$, which is to prove

    \[ \underbrace{\sqrt{2+\sqrt{2+\sqrt{2+\cdots + \sqrt{2 + \sqrt{2}}}}} }
    _\text{n+1 nested square roots} = 2\cos
    \left(  \frac{\pi}{2^{n+2}}  \right) \]

    \vspace*{0.3cm}
    \hspace*{1.2cm}
    Assume that 

    \[  \underbrace{\sqrt{2+\sqrt{2+\sqrt{2+\cdots + \sqrt{2 }}}}}_\text{
        n nested sqyare roots} = t
    \] 

    \vspace*{0.3cm}
    \hspace*{1.2cm}
    So, we only need to prove that

    \[ \sqrt{2 + t} = 2\cos{\left( \frac{\pi}{2^{n+2}} \right) } \]

    \newpage
    
    \begin{align*}
        \sqrt{2 + t}  & = 2\cos{\left( \frac{\pi}{2^{n+2}} \right) } \\
        & = 2 \cos{\left( \frac{\pi}{2^{n+1 }} \times \frac{1}{2}  \right)} \\
        2 + t & = 4 \cos^{2}{\left( \frac{\pi}{2^{n+1}} \times \frac{1}{2} \right)} \\
        2 + t & = 4 \left(  \frac{1}{2} + \frac{1}{2} \cos{\left( \frac{\pi}{2^{n+1}} \right)} \right) \\
        2 + t & = 2 + 2\cos{\left( \frac{\pi}{2^{n+1}}\right)} \\
        t & = 2\cos{\left( \frac{\pi}{2^{n+1}}\right)} \hspace*{2cm} \blacklozenge
    \end{align*}

    \vspace*{0.3cm}
    \hspace*{1.2cm}
    Since 

    \[  \underbrace{\sqrt{2+\sqrt{2+\sqrt{2+\cdots + \sqrt{2 }}}}}_\text{
        n nested sqyare roots} = t
    \] 

    \vspace*{0.3cm}
    \hspace*{1.2cm}
    According to the bease case, we know that the above equation $\blacklozenge$ is correct

    \vspace*{0.3cm}
    \hspace*{1.2cm}
    In this case, we can also say that it is correct for $n+1$

    \vspace*{0.3cm}
    $\bullet$ \textbf{In conclusion}

    \vspace*{0.3cm}
    \hspace*{1.2cm}
    So, according to the base case and inductive case, it is correct for all $n \in \mathbb{N}$ such that

    \[ \underbrace{\sqrt{2+\sqrt{2+\sqrt{2+\cdots + \sqrt{2}}}} }
    _\text{n nested square roots} = 2\cos
    \left(  \frac{\pi}{2^{n+1}}  \right) \]

\end{proof}

\newpage

\section*{Problem 6 }

\subsection*{6.1 part (a) }

\begin{proof}

    In order to prove

    \[ \sum_{k=0}^{n} 
    \begin{pmatrix}
        n \\
        k
    \end{pmatrix} = 2^n \]

    \vspace*{0.3cm}
    \hspace*{1.2cm}
    by induction, we know the range for $n$ is $ n \in \mathbb{N}_{0} $

    \vspace*{0.3cm}
    $\bullet$ \textbf{The Base Case}

    \vspace*{0.3cm}
    \hspace*{1.2cm}
    when $n=0$, it is clear that $L.H.S = R.H.S = 1$, which is correct

    \vspace*{0.3cm}
    $\bullet$ \textbf{The Inductive Case}

    \vspace*{0.3cm}
    \hspace*{1.2cm}
    We suppose the equation is correct for one $ n \in \mathbb{N}_{0} $, such that

    \[ \sum_{k=0}^{n} 
    \begin{pmatrix}
        n \\
        k
    \end{pmatrix} = 2^n \]

    
    \vspace*{0.3cm}
    \hspace*{1.2cm}
    Then we need to show that it is also correct for

    \[ \sum_{k=0}^{n+1} 
    \begin{pmatrix}
        n + 1\\
        k
    \end{pmatrix} = 2^{n+1}  \hspace*{2cm} \blacktriangle\]

    \vspace*{0.3cm}
    \hspace*{1.2cm}
    By expanding the $R.H.S$ of the above equation, we get

    \[ R.H.S = 2 \times \left[ \begin{pmatrix}
                   n  \\
                   0
    \end{pmatrix}  + \begin{pmatrix}
                         n \\
                         1
                     
                     \end{pmatrix} + \cdots + \begin{pmatrix}
                                                  n \\
                                                  n 

                     \end{pmatrix} \right] \hspace*{2cm} \blacklozenge \]

\vspace*{0.3cm}
\hspace*{1.2cm}
According to the proporty

\[ \begin{pmatrix}
       n+1 \\
       k
\end{pmatrix} = \begin{pmatrix}
                    n \\
                    k
\end{pmatrix} + \begin{pmatrix}
                    n \\
                    k-1
\end{pmatrix} : n \in \mathbb{N}_0 ; 1 \leq k \leq n \]

\vspace*{0.3cm}
\hspace*{1.2cm}
In this case, 

\vspace*{0.3cm}
\hspace*{1.2cm}
$\bullet \textbf{If n is even, then}$

\[ R.H.S = 2 \times \left[ \underbrace{\begin{pmatrix}
    n+1  \\
    1
\end{pmatrix}  + \begin{pmatrix}
          n+1 \\
          3
      
      \end{pmatrix} + \cdots + \begin{pmatrix}
                                   n+1 \\
                                   n-1 

      \end{pmatrix} }_\text{Align the first n -1 terms into pairs and apply the proporty}
      + \begin{pmatrix}
                          n \\
                          n
      \end{pmatrix} \right] \hspace*{1cm} \textit{Equation(a)} \]

\[ R.H.S = 2 \times \left[ \underbrace{\begin{pmatrix}
        n+1  \\
        2
    \end{pmatrix}  + \begin{pmatrix}
              n+1 \\
              4
          
          \end{pmatrix} + \cdots + \begin{pmatrix}
                                       n+1 \\
                                       n 
    
          \end{pmatrix} }_\text{Align the first n -1 terms into pairs and apply the proporty}
          + \begin{pmatrix}
                              n \\
                              0
          \end{pmatrix} \right] \hspace*{1cm} \textit{Equation(b)} \]

    \newpage
    \hspace*{1.2cm}
    Since we know from the proporty that
    $\begin{pmatrix}
        n \\
        n
    \end{pmatrix} = \begin{pmatrix}
        n \\
        0
    \end{pmatrix} = 1 $

    \vspace*{0.3cm}
    \hspace*{1.2cm}
    So from $\textit{Equation(a)},\textit{Equation(b)}$, we conclude that

    \[ \begin{pmatrix}
        n+1  \\
        1
    \end{pmatrix}  + \begin{pmatrix}
              n+1 \\
              3
          
          \end{pmatrix} + \cdots + \begin{pmatrix}
                                       n+1 \\
                                       n-1 
    
          \end{pmatrix} 
            =  \begin{pmatrix}
            n+1  \\
            2
        \end{pmatrix}  + \begin{pmatrix}
                  n+1 \\
                  4
              
              \end{pmatrix} + \cdots + \begin{pmatrix}
                                           n+1 \\
                                           n 
        
             
              \end{pmatrix} \]
    
    \vspace*{0.3cm}
    \hspace*{1.2cm}
    Now apply $\textit{Equation(a)} + \textit{Equation(b)} $, we get

    \[ R.H.S = \begin{pmatrix}
        n+1  \\
        1
    \end{pmatrix}  + \begin{pmatrix}
              n+1 \\
              2
          
          \end{pmatrix} +  \begin{pmatrix}
                                       n+1 \\
                                       3 
    
          \end{pmatrix} 
           + \cdots + \begin{pmatrix}
                  n+1 \\
                  n-1
              
              \end{pmatrix} + \begin{pmatrix}
                                           n+1 \\
                                           n 
        
             
              \end{pmatrix} + \begin{pmatrix}
                n \\
                0
              \end{pmatrix} + \begin{pmatrix}
                n \\
                n
              \end{pmatrix}
              \]
    
    \vspace*{0.3cm}
    \hspace*{1.2cm}
    Notice that for $n+1$, it also holds that $\begin{pmatrix}
        n+1 \\
        n+1
    \end{pmatrix} = \begin{pmatrix}
        n+1 \\
        0
    \end{pmatrix} = 1 $

    \vspace*{0.3cm}
    \hspace*{1.2cm}
    In this case,

    
       \[ R.H.S   = \begin{pmatrix}
            n+1  \\
            1
        \end{pmatrix}  + \begin{pmatrix}
                  n+1 \\
                  2
              
              \end{pmatrix} +  \begin{pmatrix}
                                           n+1 \\
                                           3 
        
              \end{pmatrix} 
               + \cdots + \begin{pmatrix}
                      n+1 \\
                      n-1
                  
                  \end{pmatrix} + \begin{pmatrix}
                                               n+1 \\
                                               n 
            
                 
                  \end{pmatrix} + \begin{pmatrix}
                    n+1 \\
                    0
                  \end{pmatrix} + \begin{pmatrix}
                    n+1 \\
                    n+1
                  \end{pmatrix} \\ \]

    \vspace*{0.3cm}
    \hspace*{1.2cm}
    Now by the proporty, it is pretty clear that

    \[ R.H.S = \sum_{k=0}^{n+1} \begin{pmatrix}
        n+1 \\
        k
    \end{pmatrix} \]

    \vspace*{0.3cm}
    \hspace*{1.2cm}
    Compare to equation $\blacktriangle$, we find that $L.H.S = R.H.S$, that is what we need to prove

    \vspace*{0.3cm}
    \hspace*{1.2cm}
    So the equality is true for even $n$

    \vspace*{0.3cm}
    \hspace*{1.2cm}
    $\bullet \textbf{If n is odd, then}$

    \[ R.H.S = 2 \times \left[ \begin{pmatrix}
        n+1  \\
        1
    \end{pmatrix}  + \begin{pmatrix}
              n+1 \\
              3
          
          \end{pmatrix} + \cdots + \begin{pmatrix}
                                       n+1 \\
                                       n-2 
    
          \end{pmatrix} 
          + \begin{pmatrix}
                              n \\
                              n
          \end{pmatrix} \right]  \]

    \vspace*{0.3cm}
    \hspace*{1.2cm}
    Since we know that from even cases,

    \[ \begin{pmatrix}
        n+1  \\
        1
    \end{pmatrix}  + \begin{pmatrix}
              n+1 \\
              3
          
          \end{pmatrix} + \cdots + \begin{pmatrix}
                                       n+1 \\
                                       n-1 
    
          \end{pmatrix} 
            =  \begin{pmatrix}
            n+1  \\
            2
        \end{pmatrix}  + \begin{pmatrix}
                  n+1 \\
                  4
              
              \end{pmatrix} + \cdots + \begin{pmatrix}
                                           n+1 \\
                                           n 
        
             
              \end{pmatrix} \]

    \vspace*{0.3cm}
    \hspace*{1.2cm}
    So, it is obvious that

    \[ R.H.S   = \begin{pmatrix}
        n+1  \\
        1
    \end{pmatrix}  + \begin{pmatrix}
              n+1 \\
              2
          
          \end{pmatrix} +  \begin{pmatrix}
                                       n+1 \\
                                       3 
    
          \end{pmatrix} 
           + \cdots + \begin{pmatrix}
                  n+1 \\
                  n-1
              
              \end{pmatrix} + \begin{pmatrix}
                                           n+1 \\
                                           n 
        
             
              \end{pmatrix} + \begin{pmatrix}
                n+1 \\
                0
              \end{pmatrix} + \begin{pmatrix}
                n+1 \\
                n+1
              \end{pmatrix} \\ \]

    \[ R.H.S = \sum_{k=0}^{n+1} \begin{pmatrix}
                n+1 \\
                k
            \end{pmatrix} \]

    
    \vspace*{0.3cm}
    \hspace*{1.2cm}
    Then back to equality $\blacktriangle$,it is clear that $L.H.S = R.H.S$

    \vspace*{0.3cm}
    \hspace*{1.2cm}
    So the equality is true for odd $n$

    
    \vspace*{0.4cm}
    $\bullet \textbf{In Conclusion}$

    \vspace*{0.3cm}
    \hspace*{1.2cm}
    So according to the base case and inductive case, $\forall n \in \mathbb{N}_{0} , 1 \leq k \leq n$

    \[ \sum_{k=0}^{n} 
    \begin{pmatrix}
        n \\
        k
    \end{pmatrix} = 2^n \hspace*{0.3cm} \text{is true. }\]


    

        
    



    
              

    
    


              

\end{proof}






\end{document}