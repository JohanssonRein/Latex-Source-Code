\documentclass[12pt]{article}
\usepackage{graphicx} % Required for inserting images
\usepackage{amsmath, amssymb, amsthm}
\usepackage{setspace, lipsum}
\usepackage[margin=2cm, headheight=15pt]{geometry}
\fontsize{24}{16}\selectfont
\setlength{\parindent}{0pt}
\title{Linear Algebra}
\author{Review Material 1}
\date{2023 6}
\usepackage{esdiff}
\usepackage{fixdif}
\renewcommand{\qedsymbol}{$\blacksquare$}


\renewcommand{\vec}[1]{\boldsymbol{#1}}
\DeclareMathOperator{\Span}{span}
\usepackage{xcolor}
\DeclareMathOperator{\rank}{rank}
\DeclareMathOperator{\kernel}{Ker}
\DeclareMathOperator{\image}{Im}
\DeclareMathOperator{\intd}{d}
\DeclareMathOperator{\trace}{tr}
\DeclareMathOperator{\lcm}{lcm}



\usepackage{fancyhdr}
\fancyhf{}
\rhead{\textit{Math 248 Written Assignment 4}}
\lhead{\textit{Written by Johnson 261105766}}
\chead{}
\cfoot\thepage
\pagestyle{fancy}


\begin{document}
\doublespacing

\section*{Question 1}
We can decompose $\vec F$ into $\vec F_1$ and $\vec F_2$ where
$\vec F = \vec F_1 + \vec F_2$, like:
\[ \vec F_1 = e^x\sin(y) \vec i + e^x\cos(y) \vec j
\hspace*{0.3cm}, \vec F_2 = 3y \vec i + 
(2x-2y) \vec j\]

And by obversation, $\vec F_1$ is also a conservative field,
meaning there exists a function $V$ such that
$\vec{\nabla} V = \vec F$. Such a function is not hard to find,
the answer is $V(x,y) = e^x\sin(y) + C$ for some $C \in 
\mathbb{R}$. Then, we come back into our problem, we 
may rewrite the intergal as:
\[ \int_C \vec F \cdot d\vec s = \int_C \vec {F_1} \cdot
d \vec s + \int_C \vec {F_2} \cdot d \vec s\]
Since $\vec F_1$ is conservative, then by the theorem,
\[ \int_C \vec F_1 \cdot d \vec s = V(B) - V(A) \]
In this problem, we may parametrize the ellipse as
$\vec c(t) = \langle \cos(t) , 2\sin(t) \rangle$ where
$t$ ranges from $0$ to $2\pi$, counter-clockwise. So
it means that
\[ \int_C \vec{F_1} \cdot d \vec s = (e^{\displaystyle{\cos(t)} }\sin(2\sin(t)) + C)
\Bigg|^{2\pi}_{0} = 0\]

Then, back to $\vec F_2$, by the parametrization of
$\vec c$, we know that
\begin{align*}
\int_C \vec F_2 \cdot d \vec s &= \int_0^{2\pi} \left[ \left(6\sin(t)\right) \vec i + \left( 2\cos(t) - 4\sin(t) \right) \vec j \right] \cdot \left[ -\sin(t) \vec i + 2\cos(t) \vec j \right] dt \\
&= \int_0^{2\pi} -6\sin^2(t) + 4\cos^2(t) - 8\sin(t)\cos(t) dt \\
&= \int_0^{2\pi} 5\cos(2t) - 4\sin(2t) -1 dt \\
&= \frac{5}{2}\sin(2t)+2\cos(2t)-t \Bigg|^{2\pi}_{0}\\
&= 1 - 2\pi \hspace{0.3cm} \text{which is the final answer}
\end{align*}

\newpage
\section*{Question 2}
\begin{proof}
By the properties of the intergal, we may write the 
origonal formula as:
\[ \int_C f \nabla g + g \nabla f \cdot d \vec s \]

Then, we observe that, by $\textit{Chain Rule}$,
$\displaystyle{\int_C f\nabla g + g \nabla f \cdot d
\vec s = f\cdot g \Bigg|^b_a}$. Then, since 
$ \mathcal{P} := \vec c(a)$, $\mathcal{Q} := 
\vec c(b)$, so we say that
\[ f\cdot g \Bigg|^b_a = f(\mathcal{Q})g(\mathcal{Q}) - 
f(\mathcal{P}) g(\mathcal{P})\]

So the equation we want to prove is correct.

\end{proof}

\newpage
\section*{Question 3}
If we project the region on the $xy$ plane, then it
is a circle centered at $(0,a)$ 
with radius $a$, and its equation can
be described as $x^2 + (y-a)^2 = a^2$. Now, by using 
the polar coordinates, we may describe the region $S 
:= \lbrace (x,y) : x^2 + (y-a)^2 \leq a^2 \rbrace $ 
as $x = r\cos\theta$, $y = r\sin\theta$ where
$0 \leq r \leq 2a$ and $\displaystyle{-\frac{\pi}{2}
\leq \theta \leq \frac{\pi}{2}}$. Now, we may consider a 
map from the region $S$ to the surface we would like to 
intergrate. The function can we written as
\[ \vec{\Phi}(r,\theta) = (r\cos\theta,r\sin\theta,
\sqrt{4a^2-r^2\cos^2\theta-r^2\sin^2\theta} ) \]
Then the cross product given by $T_r \times T_{\theta}$
is:
\[ T_r \times T_{\theta} = \det
\begin{bmatrix}
i & &  j & & k \\
\cos\theta & & \sin\theta & & \displaystyle{-\frac{r}{\sqrt{4a^2-r^2}}} \\
-r\sin\theta & & r\cos\theta & & 0
\end{bmatrix} \]
\[= (r\cos^2\theta + r\sin^2\theta)\vec k
+ (\displaystyle{\frac{r^2}{\sqrt{4a^2-r^2}}\cos\theta})
\vec i + (\displaystyle{\frac{r^2}{\sqrt{4a^2-r^2}}\sin\theta})
\vec j\]

So, the norm $\Vert T_r \times T_{\theta} \Vert = 
\displaystyle{\frac{2ar}{\sqrt{4a^2-r^2}}}$. 
(Assume $a$ is positive). In this case,
the double intergal 
\begin{align*}
\displaystyle{\iint_S} 1 ds
&= \int_{\theta = -\frac{\pi}{2}}^{\theta = \frac{\pi}{2}} \int_{r=0}^{r=2a} 1 \times 
\frac{2ar}{\sqrt{4a^2-r^2}} drd\theta \\
& \\
&= \pi \int_{0}^{2a} \displaystyle{\frac{2ar}{\sqrt{4a^2-r^2}}dr}\\
&= -a\pi \int_{2a}^{0} \frac{1}{\sqrt{u}} du \hspace*{0.3cm} By \hspace*{0.2cm} \textit{u-substitution} \\
&= \displaystyle{2a\pi u^{\frac{1}{2}}\Bigg|^{2a}_{0}} \\
&= 2\sqrt{2}a^{\frac{3}{2}}\pi \hspace*{0.3cm}, \text{which is the final answer}.
\end{align*}

\newpage
\section*{Question 4}
By our parametrization $\vec{\Phi}$, we know that the 
area at the given surface is given by:
\[ \iint_D 1 \cdot \Vert T_u  \times T_v \Vert dS \]
In this case, the intergration part is given by 
$ 0<u<2\pi$ and $0 < v < 2\pi$. Now, the cross
product can be calculated as:
\[ T_u \times T_v = \det
\begin{bmatrix}
\vec i && \vec j && \vec k \\
-r\sin u \cos v && -r \sin u \sin v && r\cos u \\
-(a+r\cos u) \sin v && (a+r\cos u) \cos v && 0
\end{bmatrix}
\]
\[ =  r(a+r\cos u) \cdot \Bigg[ -\cos u \cos v \vec i -
\cos u \sin v \vec j + \sin u \vec k \Bigg]\]

Then, by calculating the norm of it, we get
$ \Vert T_u \times T_v \Vert = r(a+r\cos u) $. In this case,
the surface intergal can be written as:
\[ \iint_D 1 \cdot \Vert T_u \times T_v \Vert dS = 
\int_0^{2\pi} \int_0^{2\pi} r(a+r\cos u) dudv\]
Since this intergal is independent of $v$, we may multiply
$2\pi$ in front of it and hence we get
$\displaystyle{2\pi \int_0^{2\pi} ar + r^2 \cos u du}$, then
it becomes 
\[ 2\pi \Bigg[ aru + r^2\sin u\Bigg]^{2\pi}_0 = 4\pi^2 ar\]




\newpage
\section*{Question 5}

By definition, the flux is given by the surface intergal 
over the vector field $\vec F = 2x \vec i + y \vec j +
z \vec k$, and the surface $ \vec{\Phi}(u,v) = 
(u^2v,uv^2,v^3)$. The flux $\vec{\Omega}$ 
with specific orentation can be defined 
as 
\[ \vec{\Omega} = \iint_D \vec{F} \vec{\cdot} 
(\vec{T}_u \times \vec{T}_v) dudv 
\]

In this case, the cross product $\vec{T}_u \times
\vec{T}_v $ is given by
\[ \det
\begin{bmatrix}
\vec i & \vec j & \vec k \\
2uv & v^2 & 0\\
u^2 & 2uv & 3v^2
\end{bmatrix} = 3v^4 \vec i - 6uv^3 \vec j + 3u^2v^2 \vec k
\]

Then, the surface intergal can be written As
\begin{align*}
\iint_D \vec{F} \vec{\cdot} 
(\vec{T}_u \times \vec{T}_v) dudv &= \int_0^1 \int_0^1 
(2u^2v \vec i + uv^2 \vec j + v^3 \vec k) \cdot 
(3v^4 \vec i - 6uv^3 \vec j + 3u^2v^2 \vec k) dudv\\
&\\
&= \int_0^1 \int_0^1 6u^2v^5 - 6u^2v^5 + 3u^2v^5 dudv \\
&\\
&= 3\int_0^1 \int_0^1 u^2v^5 dudv\\
&\\
&= 3 \int_0^1 u^2 du \int_0^1 v^5 dv \\
&\\
&= 3 \times \frac{1}{3} \times \frac{1}{6}\\
&\\
&= \frac{1}{6}, \hspace*{0.2cm} \text{which is the final answer}.
\end{align*}

\newpage
\section*{Question 6}
In this problem, the surface intergral in the vector field
$\vec F = z^2 \vec k $ can be described as
\[ \iint_{\vec{\phi}} \vec{F} \vec{\cdot} dS 
= \iint_D \vec{F} \vec{\cdot} (T_u \times T_v) dudv \]

In this problem, the function $\vec{\Omega}(\theta,\phi)$ 
can  be parametrized as $\vec{\Omega}(\theta, \phi) = 
(a \sin{\phi}\cos{\theta} , a\sin{\phi}\sin{\theta} ,
a\cos{\phi})$. Where we know that
$ \displaystyle{0 \leq \theta \leq \frac{\pi}{2}
}$ and $\displaystyle{0 \leq \phi \leq 
\frac{\pi}{2}}$. Now, the cross product of $T_{\theta}$ and $T_{\phi}$ is
given by

\[ T_{\theta} \times T_{\phi} = \det
\begin{bmatrix}
\vec i & &\vec j & & \vec k \\
-a\sin{\phi}\sin{\theta} && a\sin{\phi}\cos{\theta}&&0\\
a\cos{\phi}\cos{\theta}&&a\cos{\phi}\sin{\theta}&&-a\sin{\phi}
\end{bmatrix}
\]
\[ = (-a^2\sin^2{\phi}\cos{\theta}) \vec i +
(-a^2\sin^2{\phi}\sin{\theta})\vec j + (-a^2\sin
{\phi}\cos{\phi})\vec k \]

Then, by using $z = a\cos{\phi}$, we get the surface
intergal as:
\begin{align*}
\iint_D \vec{F} \vec{\cdot} (T_u \times T_v) dudv 
&= \int_{0}^{\frac{\pi}{2}}\int_0^{\frac{\pi}{2}}
(a^2\cos^2{\phi})\vec k \vec{\cdot} (-a^2\sin^2{\phi}\cos{\theta} \vec i - a^2\sin^2{\phi}\sin{\theta}\vec j - a^2 \sin{\phi}\cos{\phi} \vec k) d \theta d \phi \\
&\\
&= \int_{0}^{\frac{\pi}{2}}\int_0^{\frac{\pi}{2}} -a^4\sin{\phi}\cos^3{\phi} d\theta d \phi \\
&\\
&= -\frac{a^4\pi}{2} \int_0^{\frac{\pi}{2}} \sin{\phi}\cos^3{\phi} d\phi\\
&\\
&= \frac{a^4\pi}{2}\int_1^0 u^3 du \hspace*{0.2cm} \textit{By u-Substitution} \\
&\\
&= -\frac{a^4 \pi}{8}
\end{align*}
Since the norm $\vec n$ of the surface is of the "same" direction as the vector field, 
i.e $\vec F \cdot \vec n > 0$. Hence
the value should be positive, so the fulx is
$\displaystyle{\frac{a^4\pi}{2}}$.


\end{document}