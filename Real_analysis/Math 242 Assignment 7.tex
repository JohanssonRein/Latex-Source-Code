\documentclass[12pt]{article}
\usepackage{graphicx} % Required for inserting images
\usepackage{amsmath, amssymb, amsthm}
\usepackage{setspace, lipsum}
\usepackage[margin=2cm, headheight=15pt]{geometry}
\fontsize{24}{16}\selectfont
\setlength{\parindent}{0pt}
\title{Linear Algebra}
\author{Review Material 1}
\date{2023 6}
\usepackage{esdiff}
\usepackage{fixdif}
\renewcommand{\qedsymbol}{$\blacksquare$}


\renewcommand{\vec}[1]{\boldsymbol{#1}}
\DeclareMathOperator{\Span}{span}
\usepackage{xcolor}
\DeclareMathOperator{\rank}{rank}
\DeclareMathOperator{\kernel}{Ker}
\DeclareMathOperator{\image}{Im}
\DeclareMathOperator{\intd}{d}
\DeclareMathOperator{\trace}{tr}
\DeclareMathOperator{\lcm}{lcm}



\usepackage{fancyhdr}
\fancyhf{}
\rhead{\textit{Math 242 Written Assignment 7}}
\lhead{\textit{Written by Johnson 261105766}}
\chead{}
\cfoot\thepage
\pagestyle{fancy}


\begin{document}
\doublespacing

\section*{Question 1}

The sequence $(x_n)$ which satisifies the properties 
as mentioned is: 
\[ x_n = n + \frac{1}{n} \]

\begin{center}
    $\textbf{Now let's prove it}$
\end{center}

\begin{proof}
    We will show that $\forall \epsilon > 0$, $\exists N
    \in \mathbb{N}$, s.t $\forall n \geq N$,
    $\vert x_{n+1} - x_n \vert < \epsilon$.

    \vspace*{0.4cm}
    It is not hard to observe that $\displaystyle{
    \vert x_{n+1} - x_n \vert = \Bigg| \left( n +1+
    \frac{1}{n+1} \right) - \left(n + \frac{1}{n}\right)}
    \Bigg| = \Bigg| \frac{1}{n+1} - \frac{1}{n} + 1 \Bigg|$.

    \vspace*{0.5cm}
    Then, we can do further simplification,
    $\displaystyle{\vert x_{n+1} - x_n \vert = \Bigg| \frac{-1}{n(n+1)}
    + 1 \Bigg| } $. By $\textit{Triangle Inequality}$,
    
    \begin{align*}
    \vert x_{n+1} - x_n \vert &\leq 
    \Bigg| \frac{-1}{n(n+1)}  \Bigg|+  \vert 1 \vert \\
    &= \displaystyle{\frac{1}{n(n+1)} + 1}
    \end{align*}

    Since $n$ is always a positive integer, So
    $n+1 > n$, conversely $\displaystyle{\frac{1}{n+1}
    < \frac{1}{n}}$. So we get
    \[ \vert x_{n+1} - x_n \vert <
    \frac{1}{n^2} + 1\]

    In this case, if we slove for $\displaystyle{
    \frac{1}{n^2} + 1 < \delta}$, where $\delta >1$,
    it's the same as sloving $\displaystyle\frac{1}{n^2} <
    \epsilon$. where $\epsilon > 0$.

    So, we get when $n \geq \displaystyle{
    \sqrt{\frac{1}{\epsilon}}}$, it is always true that
    $\displaystyle{\vert x_{n+1} - x_{n} \vert 
    < \frac{1}{n^2} + 1 < \epsilon + 1}$. 

    \vspace*{0.3cm}
    $\textbf{However}$, as we can see, $(x_n)$ does
    not converge, because it is a strictly increasing
    sequence when $n \geq 1$, and $\lim(x_n)$ does not
    exist. (infinity)

    \vspace*{0.6cm}
    $\textbf{So, according to the judgements above,
    this sequence satisities those properties}$.

\end{proof}

\newpage
\section*{Question 2}

\subsection*{Part (a) :}
\begin{proof}
Since we know that $x_1 > 0$, then the base case is 
automatically correct.Now we assume that the statement
is true for a fixed integer $n$. Then we will prove that
it still holds for $n+1$. Since we know $x_n >0$ by our
assumption, and according to the definition of this
sequence, $x_{n+1} = \displaystyle{\frac{1}{2 + x_n}}$,
which means that $x_{n+1}$ must also be positive. So
the inductive step is also proved. Hence, 
$\textit{WLOG}$, we say that $x_n > 0$ for all $\mathbb{N}$.
\end{proof}

\subsection*{Part (b) :}
\begin{proof}
To show tht $(x_n)$ is contractive, we need to show that
$\forall n \in \mathbb{N}$, $\exists c \in (0,1)$, s.t
\[ \vert x_{n+2} - x_{n+1} \vert \leq c\cdot
\vert x_{n+1} - x_n \vert \]

By the formula $x_{n+1} = \displaystyle{\frac{1}{2+x_n}}$,
we get $\vert x_{n+2} - x_{n+1} \vert = 
\Bigg| \displaystyle{\frac{1}{2+x_{n+1}} - \frac{1}{2+x_n}}
\Bigg| $. Then, by further simplification, we get

\[\vert x_{n+2} - x_{n+1} \vert = 
\Bigg| \displaystyle{\frac{x_n - x_{n+1}}{(2+x_{n+1})(2+x_n)}
\Bigg| = \frac{1}{\vert (2+x_{n+1})(2+x_n)\vert } \cdot
\big| x_{n+1} - x_n \big| }\]

We see that the term $ 
\Bigg| \displaystyle{\frac{1}{(2+x_{n+1})(2+x_n)}
\Bigg|}$ is always between $0$ and $1$ by the property, and
it is constant befined by a given sequence, also note that
$x_n$ is always positive for any $n \in \mathbb{N}$, which
means $\displaystyle{\frac{1}{(2+x_{n+1})(2+x_n)}} < 
\frac{1}{2\times 2} < 1$. So by the property, we conclude that

\[ \vert x_{n+2} - x_{n+1} \vert \leq c\cdot
\vert x_{n+1} - x_n \vert \]
\end{proof}

\subsection*{Part (c) :}
\begin{proof}
By the property, we know that $\lim(x_{n+1}) = \lim(x_n)$,
then if we use the relations between $x_{n+1}$ and $x_n$, 
we get 

\[ \lim(x_{n+1}) = \lim(\frac{1}{2+x_n})\]

Then by limit laws, we can simplify as:
\[ \lim(x_{n+1}) = \frac{1}{2 + \lim(x_n)} \]

As we mentioned, $\lim(x_{n+1}) = \lim(x_n)$, so we may
let $\lim(x_n) = u$, then solve for the equation
\[ u = \frac{1}{2+u} \]

Then we get $u = -1 \pm \sqrt{2}$, but note that by definition,
$x_n$ is always positive, so the only solution inserting
$u = -1 + \sqrt{2}$, i.e $\lim(x_n) = -1 + \sqrt{2}$.
\end{proof}





\end{document}