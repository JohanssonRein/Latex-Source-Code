\documentclass[12pt]{article}
\usepackage{graphicx} % Required for inserting images
\usepackage{amsmath, amssymb, amsthm}
\usepackage{setspace, lipsum}
\usepackage[margin=2cm, headheight=15pt]{geometry}
\fontsize{24}{16}\selectfont
\setlength{\parindent}{0pt}
\title{Linear Algebra}
\author{Review Material 1}
\date{2023 6}
\usepackage{esdiff}
\usepackage{fixdif}
\renewcommand{\qedsymbol}{$\blacksquare$}


\renewcommand{\vec}[1]{\boldsymbol{#1}}
\DeclareMathOperator{\Span}{span}
\usepackage{xcolor}
\DeclareMathOperator{\rank}{rank}
\DeclareMathOperator{\kernel}{Ker}
\DeclareMathOperator{\image}{Im}
\DeclareMathOperator{\intd}{d}
\DeclareMathOperator{\trace}{tr}
\DeclareMathOperator{\lcm}{lcm}
\DeclareMathOperator{\pt}{\partial}
\newcommand{\integral}[4]{\int\limits^{#1}_{#2}\int\limits^{#3}_{#4}}



\usepackage{fancyhdr}
\fancyhf{}
\rhead{\textit{Math 248 Written Assignment 3}}
\lhead{\textit{Written by Johnson 261105766}}
\chead{}
\cfoot\thepage
\pagestyle{fancy}


\begin{document}
\doublespacing

\section*{Question 1}

\subsection*{Part (a) :}
Since $ \vec {F} = (x^2 + y^2) \vec i - 2xy \vec j$, 
note that there are no "holds" on 
variables $x,y$ so $x,y \in \mathbb{R}$ and the 
domain of $\vec F$ is $\mathbb{R}^2$.

\subsection*{Part (b) :}
If $\vec F$ is conservative, then there exists
 $\phi(x,y) \in U$ such that 
$\displaystyle{\frac{\partial \phi}{\partial x} =
x^2 + y^2}$ and $\displaystyle
{\frac{\partial \phi}{\partial y} =-2xy}$. Then

\begin{center}
    $\displaystyle
    {\frac{\partial \phi}{\partial x \partial y} =
    2y}$ , $\displaystyle
    {\frac{\partial \phi}{\partial y \partial x} 
    =-2}$
\end{center}

Suppose such a function $\phi(x,y)$ exists, and both
$\phi_{xy}$ and $\phi_{yx}$ are functions in $C^1$,
then $\phi$ itself must have continuous first and 
second order partial derivatives. But clearly
$\displaystyle
    {\frac{\partial \phi}{\partial x \partial y} 
     \neq \frac{\partial \phi}{\partial y \partial x} 
    }$.
Which violates the equality of mixed partials
$(\textit{Schwarz Theorem})$. Thus, this cannot because
the case, such a function $\phi(x,y)$ doesn't exist.
So, $\vec F $ is not a conservative vector field.


\newpage
\section*{Question 2}
\subsection*{Part (a) :}

\begin{center}
    Note that 
    $\vec F = \displaystyle{(xy - \sin z) \vec i
    + \left( \frac{x^2}{2} - \frac{e^y}{z} \right)
    \vec j + \left( \frac{e^y}{z^2} - x \cos z \right)
    \vec k}$
\end{center}

Since the denominator cannot be zero, which implies
$ z \neq 0$ and $z^2 \neq 0$. So $z \neq 0$.

In this case, the domain $U$ is $\mathbb{R}^3 \backslash
\lbrace z = 0 \rbrace $.

\subsection*{Part (b) :}
By definition, 
\[ \vec{\nabla} \times \vec F = \displaystyle{
    \det
\begin{bmatrix}
    \vec i & \vec j & \vec k \\
     & & \\
    \displaystyle{\frac{\pt}{\pt x}} & \displaystyle{\frac{\pt}{\pt y}} & \displaystyle{\frac{\pt}{\pt z}} \\
     & & \\
    xy - \sin z & \displaystyle{\frac{x^2}{2} - \frac{e^y}{z} }& \displaystyle{\frac{e^y}{z^2} - x\cos z}
\end{bmatrix} } \]
\[ = \left[\frac{\pt}{\pt y}\left( \frac{e^y}{z^2} -x\cos z
\right) - \frac{\pt}{\pt z}\left( \frac{x^2}{2} -
\frac{e^y}{z}\right) \right] \vec i 
+ \left[ \frac{\pt}{\pt x}\left( \frac{e^y}{z^2} -x\cos z
\right) - \frac{\pt}{\pt z}\left( xy -
\sin z\right)\right] \vec j + \]
\[ \left[
\frac{\pt}{\pt x}\left( \frac{x^2}{2} - \frac{e^y}{z}
\right) - \frac{\pt}{\pt y}\left( xy - \sin z\right)
\right] \vec k \]

\[ = \left( \frac{e^y}{z^2} - \frac{e^y}{z^2} \right)
\vec i + \left( -\cos z + \cos z \right) \vec j +
\left( x - x \right) \vec k = 0\]

So $\vec{\nabla} \times \vec{F} = 0$, according to 
the calculation.

\subsection*{Part (c) :}

To find the function $V(x,y,z)$ such that $\vec{F}
= \vec{\nabla} V$, we can do the following:


For the first term, we know that $\displaystyle{
    \frac{\pt V}{\pt x} = xy - \sin z
}$, so $V = \displaystyle{\frac{1}{2}x^2y - x\sin z
+ h(y,z)}$, where $h(y,z)$ is a function only 
depend on variables $y,z$.

We can do the same on the second term, we find that
$\displaystyle{\frac{\pt V}{\pt y} = 
\frac{x^2}{2} - \frac{e^y}{z}}$, and compare to the 
$V$ we obtained in the first term, we find that
\[ \frac{\pt V}{\pt y} = \frac{\pt
\left(\frac{1}{2}x^2 y - x\sin z\right)}{\pt y} +
\frac{\pt h(y,z)}{\pt y} = \frac{1}{2}x^2 + 
\frac{\pt h(y,z)}{\pt y} = \frac{1}{2}x^2 - 
\frac{e^y}{z}\]

So $\displaystyle{\frac{\pt h(y,z)}{\pt y} = -\frac{e^y}{z}}
$, which means that $h(y,z) = \displaystyle{
    -\frac{e^y}{z} + p(z)
}$.

Now, do the same operation on the third term, we get

\[ \frac{\pt V}{\pt z} = \frac{\pt \left(
\frac{1}{2}x^2 y - x\sin z - \frac{e^y}{z} + p(z) \right)}{\pt z} 
= -x\cos z + \frac{e^y}{z^2} + \frac{ \pt p(z)}{\pt z}
= \frac{e^y}{z^2} - x\cos z\]

So, we get $\displaystyle{\frac{\pt p(z)}{\pt z} = 
0}$, so $p(z) = C$, where $C$ is a constant.

\vspace*{0.2cm}
Then, we find that $V(x,y,z) = \displaystyle{\frac{1}{2}
x^2y - x\sin z - \frac{e^y}{z} + C}$.







\newpage
\section*{Question 3}
By definition, we know that

\[ \vec{\nabla} f = \left(
    \frac{\pt f}{\pt x},\frac{\pt f}{\pt y},
    \frac{\pt f}{\pt z}
\right) \hspace*{0.2cm} 
\vec{\nabla} g = \left(
    \frac{\pt g}{\pt x},\frac{\pt g}{\pt y},
    \frac{\pt g}{\pt z}
\right)\]

\begin{align*} 
\vec{\nabla f} \times \vec{\nabla g} & = \det
\begin{bmatrix}
    \vec i & \vec j & \vec k \\
    & & \\
    \displaystyle{\frac{\pt f}{\pt x}} & \displaystyle{\frac{\pt f}{\pt y}} & \displaystyle{\frac{\pt f}{\pt z}} \\
    & & \\
    \displaystyle{\frac{\pt g}{\pt x}} & \displaystyle{\frac{\pt g}{\pt y}} & \displaystyle{\frac{\pt g}{\pt z}}
\end{bmatrix} \\
&\\
& = \displaystyle{\left( \frac{\pt f \pt g}{\pt y \pt z} - \frac{\pt f \pt g}{\pt z \pt y}\right)\vec i} 
- \displaystyle{\left( \frac{\pt f \pt g}{\pt x \pt z} - \frac{\pt f \pt g}{\pt z \pt x}\right)\vec j}
+ \displaystyle{\left( \frac{\pt f \pt g}{\pt x \pt y }-\frac{\pt f \pt g}{\pt y \pt x} \right)\vec z}
\end{align*}

So, in this case, 
\[\vec{\nabla} \cdot (\vec{\nabla f} \times
\vec{\nabla g}) = \frac{\pt}{\pt x} \left( \frac{\pt f \pt g}{\pt y \pt z} - \frac{\pt f \pt g}{\pt z \pt y}\right)\vec i
+ \frac{\pt}{\pt y} \left( \frac{\pt f \pt g}{\pt x \pt z} - \frac{\pt f \pt g}{\pt z \pt x}\right)\vec j
+\frac{\pt }{\pt z} \left( \frac{\pt f \pt g}{\pt x \pt y }-\frac{\pt f \pt g}{\pt y \pt x} \right)\vec z\]

Clearly, all three terms are zero. So we say that 
$\vec{\nabla} \cdot (\vec{\nabla f} \times
\vec{\nabla g}) = 0 $.

\newpage
\section*{Question 4}

\subsection*{Part (a) :}
By showing that $L(t)$ is monotone increasing, we may
show that $ L'(t) \geq 0 $. By differentiate both sides
of $L(t) = V(\vec c(t))$, we get 
\[ L'(t) = \frac{d}{dt}V(\vec c(t)) = 
\vec{\nabla} V \cdot \frac{dc}{dt}\]
By knowing that $F = \vec{\nabla}V$, we can say that
$\displaystyle{L'(t) = F \cdot \frac{dc}{dt}}$. If we
denote the angle between $F$ and $\displaystyle{\frac{dc}{dt}}$
to be $\theta$, then
\[ L'(t) = \Vert F \Vert
\Bigg |\Bigg|\frac{dc}{dt} \Bigg|\Bigg|\cos{\theta} \]
Then, since $F$ is a conservative vector field and
$\vec c(t)$ is a flow line, we know that
$F(\vec c(t)) = \vec c'(t)$, which is saying that,
the angle between $F$ and $\displaystyle{\frac{dc}{dt}}$
is $0$, i.e they are parallel. So in this case
$\cos{\theta} = 1$. Then, we find that
\[L'(t) = \Vert F \Vert \displaystyle{
    \Bigg| \Bigg| \frac{dc}{dt} \Bigg| \Bigg|
} \hspace*{0.2cm} \text{which is of course,
non-negative.}\]
So, we have shown that $L'(t) \geq 0$, saying that
$L(t)$ is monotone increasing.



\subsection*{Part (b) :}

Since $\vec{F} = x \vec i + 2y \vec j + 5z \vec k $,
we need to find $V(x,y,z)$ such that $\vec{F} = 
\vec{\nabla} V$.
We can see that $x,y,z$ are the only dependent
variables in $\vec i, \vec j, \vec k$ terms, so We
can just intergrate each part and get the result.
Say $V(x,y,z) = \displaystyle{ \int x dx + 
\int 2y dy + \int 5z dz }$, so we get
\[ V(x,y,z) = \frac{1}{2}x^2 + y^2 + \frac{5}{2}z^2
+ C \].
\subsection*{Part (c) :}
Recall the definition, the flow line $\vec c(t)$ 
satisifies $\vec c^{\prime}(t) = \vec{F}(\vec c(t))$.
In this problem, $\vec{F} = P\vec i + Q\vec j + R\vec k$
where $P = x$, $Q = 2y$, $R = 5z$. The flow line
$\vec c(t)  = x(t) \vec i + y(t) \vec j + z(t) \vec k$
also satisifies:
\[\begin{cases}
    x^{\prime}(t) = P(x(t),y(t),z(t)) \\
    y^{\prime}(t) = Q(x(t),y(t),z(t)) \\
    z^{\prime}(t) = R(x(t),y(t),z(t)) 
\end{cases}\]
by plug in those values, we get
\[\begin{cases}
    x^{\prime}(t) = x(t) \\
    y^{\prime}(t) = 2y(t) \\
    z^{\prime}(t) = 5z(t)
\end{cases} \hspace*{0.2cm} \Longrightarrow
\hspace*{0.2cm}
\begin{cases}
    x(t) = e^t + C_1 \\
    y(t) = e^{2t} + C_2 \\
    z(t) = e^{5t} + C_3
\end{cases}\]

And we know that $\vec{c}(0) = (1,15,7)$, then solve for
$C_1,C_2,C_3$ we get $C_1 = 0$, $C_2 = 14$, 
$C_3 = 6$. 

So the folw line is $\vec{c}(t) = \left(
    e^t , e^{2t} + 14, e^{5t} + 6
\right)$.


\subsection*{Part (d) :}
As we have shown in part (a), $\Vert \vec F \Vert = 
\sqrt{x^2+4y^2+25z^2}$, $\displaystyle{
    \Bigg| \Bigg| \frac{dc}{dt} \Bigg|\Bigg|
 = \sqrt{e^{2t}+4e^{2t}+25e^{10t}}}$, of course
their multiple is non-negative.



\newpage
\section*{Question 5}

In this problem, we may seperate the area into smaller
parts in order to simplify our integration. 
Let's define the following sub-areas:
\[\begin{cases}
    D_1 := \lbrace (x,y) \Big \vert 0 \leq x \leq 4, y \leq x \rbrace \\
    D_2 := \lbrace (x,y) \Big \vert 4 \leq x \leq 8, 0 \leq y \leq 2 \rbrace \\
    D_3 := \lbrace (x,y) \Big \vert 4 \leq x \leq 8, 2 \leq y \leq \displaystyle{\frac{16}{x} }\rbrace
\end{cases}\]

Clearly, $D = D_1 + D_2 + D_3$. So by the definition,
\[ \iint_D x^2 dxdy = \iint_{D_1} x^2 dxdy +
\iint_{D_2} x^2 dxdy + \iint_{D_3} x^2 dxdy\]

Note that we may need to change the integration variable,
so we do the following. This problem can also solved by using Fubini's
Theorem, but in this case both methods are almost the same.

\[\begin{cases}
    D_1 := \lbrace (x,y) \Big \vert 0 \leq y \leq 4, y \leq x \leq 4 \rbrace \\
    D_2 := \lbrace (x,y) \Big \vert 0 \leq y \leq 2, 4 \leq x \leq 8 \rbrace \\
    D_3 := \lbrace (x,y) \Big \vert 2 \leq y \leq 4, 4 \leq x \leq \displaystyle{\frac{16}{y} }\rbrace
\end{cases}\]

Then, we can calculate the integrals seperately.
\[ \iint_{D_1} x^2 dxdy = 
\int_{y = 0}^{y = 4}\int_{x = y}^{x = 4} x^2 dxdy
= \frac{1}{3}\int_{y=0}^{y=4} 64 - y^3 dy
= 64\]
\[ \iint_{D_2} x^2 dxdy = \int_{y=0}^{y=2} 1 dy 
\int_{x=4}^{x=8} x^2 dx = 2\int_{x=4}^{x=8}x^2 dx
= \frac{896}{3}\]
\[ \iint_{D_3} x^2 dxdy = \int_{y=2}^{y=4} 
\int_{x = 4}^{x = \frac{16}{y}}x^2 dxdy =
\frac{1}{3}\int_{y=2}^{y=4} \left(\frac{16}{y}\right)^3 - 64 dy
= \frac{256}{3}\] 

By summing up all the integrals we have calculated,
we get $\displaystyle{\iint_D x^2 dxdy} = \displaystyle{64 + 
\frac{896}{3}+\frac{256}{3}} = 448$.

\newpage
\section*{Question 6}
By observation, we can intergate by changing the integration
variables. Namely say $ u = x^2 - y^2 $, $v = xy$.
And it is easy to know the range of $u,v$, where
$ 1 \leq u \leq 9$ and $2 \leq v \leq 4$. Then, by
change of variable matrix, we need to find the 
functions $x(u,v)$ and $y(u,v)$. As we can see, we have the
equality $\displaystyle{D(T)^{-1} = (D(T))^{-1}}$, 
which is saying that
\[ \Bigg| \frac{\pt(x,y)}{\pt(u,v)}\Bigg| = 
\frac{1}{\Bigg|\displaystyle{\frac{\pt(u,v)}{\pt(x,y)}\Bigg|}}\]
So, $\displaystyle{\Bigg| \frac{\pt(x,y)}{\pt(u,v)} \Bigg|}
= \frac{1}{ \det
\begin{bmatrix}
u_x & u_y \\
v_x & v_y
\end{bmatrix}} = 
\frac{1}{ \det
    \begin{bmatrix}
        2x & -2y \\
        y & x
    \end{bmatrix}}
= \displaystyle{\frac{1}{2x^2 + 2y^2}}$. In this case,
we get
\[ \iint_D \left( x^2 + y^2 \right) dxdy 
= \iint_{D^*} \left( x^2 + y^2 \right) \cdot
\Bigg| \frac{\pt(x,y)}{\pt(u,v)} \Bigg| dudv
= \iint_{D^*}  dudv\]
Then, by $\textit{Fubini's Theorem}$, when we
plug in the range of $u,v$, we get
\[ \iint_{D^*} dudv = \int_{u=1}^{u=9}du \cdot
\int_{v=2}^{v=4} dv \]
Since both integrals are independent of their 
integration variables, then
\[ \iint_{D^*} dudv = (9-1)\times(4-2) = 16 \]
So we conclude that $\displaystyle{
    \iint_{D}\left(x^2 + y^2\right)dxdy = 16
}$, with respect to the area bounded by $x,y$.


\newpage
\section*{Question 7}

By sketchingthe region, we found that the shape is
like an "ice cream cone". The equation
$x^2 + y^2 + z^2 = a^2$ is a sphere with radius $a$,
and the equation $x^2 + y^2 - (\tan \alpha)^2 z^2 = 0$
is a cone, and we can calculate the "slope" of that cone,
which is given by the angle $\alpha$. Then, by changing
the variables, let $x = r\cos \theta$, $
y = r \sin \theta $, $z = z$. Then the 
determinant of the change of
variable matrix is given by:

\[ \Bigg| \frac{\pt(x,y,z)}{\pt(r,\theta,z)} \Bigg| 
= \det
\begin{bmatrix}
    x_r & x_{\theta} & x_z \\
    y_r & y_{\theta} & y_z \\
    z_r & z_{\theta} & z_z
\end{bmatrix} = r\]

\vspace*{0.3cm}
The range of the new variables are defined by:
\[ r \in \left[ 0,a\sin{\alpha} \right] ,
\theta \in \left[ 0 , 2\pi \right],
z \in \left[ \displaystyle{
\sqrt{\frac{x^2 + y^2}{\tan^2 \alpha}},
\sqrt{a^2-x^2-y^2}} \right]\]

And again, $x^2 + y^2 = r^2$, by doing that, the integral
can be calculated as:

\begin{align*}
 V &= \int_{\theta = 0}^{\theta = 2\pi}
\int_{r=0}^{r=a\sin \alpha} \int_{z = 
\frac{r}{\tan \alpha}}^{z = \sqrt{a^2-r^2}} r
dzdrd\theta\\
&\\
&=\int_{\theta = 0}^{\theta = 2\pi}
\int_{r=0}^{r=a\sin \alpha}
r \left( \sqrt{a^2-r^2} - \displaystyle{\frac{r}{\tan \alpha}} \right) drd\theta\\
&\\
&=\int_{\theta = 0}^{\theta = 2\pi} \displaystyle{
\left( -\frac{1}{3}(a^2-r^2)^{\frac{3}{2}}-
\frac{1}{3\tan \alpha}r^3 \right)_{r = 0}^{r = a\sin \alpha} d \theta
\hspace*{0.2cm} \textit{(By u-substitutin)}
} \\
&\\
&=\int_{\theta = 0}^{\theta = 2\pi} \displaystyle{
-\frac{1}{3}(a\cos\alpha)^3 - \frac{a^3\sin^3\alpha}{3\tan \alpha}
+\frac{1}{3}a^3 d\theta
}
= \displaystyle{\frac{2\pi a^3}{3}\left(
    1 - \frac{\sin^3 \alpha}{\tan \alpha} - \cos^3 \alpha
\right)}
\end{align*}

So, the volume of that shape is given By
$V=\displaystyle{\frac{2\pi a^3}{3}\left(
    1 - \frac{\sin^3 \alpha}{\tan \alpha} - \cos^3 \alpha
\right)}$.













\end{document}