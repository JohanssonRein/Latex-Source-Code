\documentclass[12pt]{article}
\usepackage{graphicx} % Required for inserting images
\usepackage{amsmath, amssymb, amsthm}
\usepackage{setspace, lipsum}
\usepackage[margin=2cm, headheight=15pt]{geometry}
\fontsize{24}{16}\selectfont
\setlength{\parindent}{0pt}
\title{Linear Algebra}
\author{Review Material 1}
\date{2023 6}
\usepackage{esdiff}
\usepackage{fixdif}
\renewcommand{\qedsymbol}{$\blacksquare$}


\renewcommand{\vec}[1]{\boldsymbol{#1}}
\DeclareMathOperator{\Span}{span}
\usepackage{xcolor}
\DeclareMathOperator{\rank}{rank}
\DeclareMathOperator{\kernel}{Ker}
\DeclareMathOperator{\image}{Im}
\DeclareMathOperator{\intd}{d}
\DeclareMathOperator{\trace}{tr}
\DeclareMathOperator{\lcm}{lcm}



\usepackage{fancyhdr}
\fancyhf{}
\rhead{\textit{Math 242 Written Assignment 6}}
\lhead{\textit{Written by Johnson 261105766}}
\chead{}
\cfoot\thepage
\pagestyle{fancy}


\begin{document}
\doublespacing

\section*{Question 1}

\subsection*{Part (a) :}
\begin{proof}
By Bernoulli's inequality: $\displaystyle{\left( 
1 + \frac{1}{\sqrt{n}}\right)^n \geq 1 + n \times
\frac{1}{\sqrt{n}} = 1 + \sqrt{n} > \sqrt{n}}$.
So it's true for all $n \in \mathbb{N}$.
\end{proof}

\subsection*{Part (b) :}
\begin{proof}
In order to prove $\displaystyle{\left( 
1 + \frac{1}{\sqrt{n}} \right)^2 > \sqrt[n]{n}}$,
first by taking the n-th square on both sides, 
which is equivalant as proving 
$\displaystyle{\left( 
1 + \frac{1}{\sqrt{n}} \right)^{2n} > n}$.

\begin{align*}
    \left( 
    1 + \frac{1}{\sqrt{n}} \right)^{2n} &> n \\
    \left( 1 + \frac{1}{\sqrt{n}}
    \right)^n \left( 1 + \frac{1}{\sqrt{n}}
    \right)^n &> n \\
    & \geq \left( 1 + \sqrt{n} \right) \left( 1 + \sqrt{n} \right) 
    \hspace*{0.2cm} \textit{By Bernoulli's
    inequality}\\
    & = 1 + n + 2\sqrt{n} \\
    & > n.
\end{align*}
\end{proof}
\subsection*{Part (c) :}
\begin{proof}
By definition, we need to show that $\exists N \in
\mathbb{N}$, such that $\forall n \geq N$,
$\vert a_n - L \vert < \epsilon$ for all $\epsilon
> 0$. i.e $ \vert \sqrt[n]{n} - 1 \vert < \epsilon$.
By part (b), we know that
\[\vert \sqrt[n]{n} - 1 \vert <
\displaystyle{ \Bigg| \left( 1 + \frac{1}{\sqrt{n}}
\right)^2 -1 \Bigg|} = \Bigg| 1 + \frac{1}{n} +
\frac{2}{\sqrt{n}} - 1 \Bigg| = \Bigg| 
\frac{2}{\sqrt{n}} + \frac{1}{n} \Bigg|\]
For large $n$, especially $n >1 $, we have
$ n > \sqrt{n} $, and we can get rid of absolute 
signs because they are all positive. so

\[ \Bigg| 
\frac{2}{\sqrt{n}} + \frac{1}{n} \Bigg| < 
\frac{2}{\sqrt{n}} + \frac{1}{\sqrt{n}} = \frac{3}{\sqrt{n}} < \epsilon\]

Now, solve for $n$, we have $ \displaystyle{ n >
\left( \frac{3}{\epsilon} \right)^2}$. So in this case,
we have proved that when $N = \displaystyle{ 
\left( \frac{3}{\epsilon} \right)^2}$, for all
$n \geq N$, $\vert a_n - L \vert < \epsilon$. Hence

\[ \lim \left( \sqrt[n]{n} \right) = 1 \]
\end{proof}

\newpage
\section*{Question 3}
\begin{proof}
Knowing that $0<a<b$, and the function $f(x) = \sqrt{x}$ is increasing, we
know that it is true for $\sqrt[n]{2a^n} < \sqrt[n]{a^n+b^n} < \sqrt[n]{2b^n}$
, i.e $a\sqrt[n]{2} < \sqrt[n]{a^n+b^n} < b\sqrt[n]{2}$. We know that the
sequence $(x_n) = \sqrt[n]{2}$ is convergent, because we can rewrite as
$ x_n = \displaystyle{2^{\frac{1}{n}}}$ and $\displaystyle{0< \frac{1}{n} < 1}$.
Since $a,b$ are constant, then the sequences $ a_n = a\sqrt[n]{2}$ and
$b_n = b\sqrt[n]{2}$ are also convergent, which will make the sequence
$x_n = \sqrt[n]{a^n+b^n}$ between them to be convergent. So $(x_n)$
converges.




\end{proof}





\end{document}