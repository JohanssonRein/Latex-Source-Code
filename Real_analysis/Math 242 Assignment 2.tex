\documentclass[12pt]{article}
\usepackage{graphicx} % Required for inserting images
\usepackage{amsmath, amssymb, amsthm}
\usepackage{setspace, lipsum}
\usepackage[margin=2cm, headheight=15pt]{geometry}
\fontsize{24}{16}\selectfont
\setlength{\parindent}{0pt}
\title{Linear Algebra}
\author{Review Material 1}
\date{2023 6}
\usepackage{esdiff}
\usepackage{fixdif}


\renewcommand{\vec}[1]{\boldsymbol{#1}}
\DeclareMathOperator{\Span}{span}
\usepackage{xcolor}
\DeclareMathOperator{\rank}{rank}
\DeclareMathOperator{\kernel}{Ker}
\DeclareMathOperator{\image}{Im}
\DeclareMathOperator{\intd}{d}
\DeclareMathOperator{\trace}{tr}



\usepackage{fancyhdr}
\fancyhf{}
\rhead{\textit{Math 242 Written Assignment 2}}
\lhead{\textit{Written by Johnson 261105766}}
\chead{}
\cfoot\thepage
\pagestyle{fancy}


\begin{document}
\doublespacing

\section*{Problem 2(b) (i)}

\begin{proof}
    In order to prove that $ f^{-1}(A \bigcap B) = f^{-1}
    (A) \bigcap f^{-1}(B)$, we first show that
    $ L.H.S \subseteq R.H.S $.

    \subsection*{(1)}
    $\bullet$ $L.H.S \subseteq R.H.S $
    

    Suppose $ x \in f^{-1}(A \bigcap B)$, by definition, then
    $f(x) \in A \bigcap B $, which is 
    $ f(x) \in A \land f(x) \in B $, that is
     $ x \in f^{-1}(A) \land x \in f^{-1}(B)$, the same as
     $ x \in f^{-1}(A) \bigcap x \in f^{-1}(B) $.
     So $L.H.S \subseteq R.H.S $

    
    \subsection*{(2)}
    $\bullet$ $R.H.S \subseteq L.H.S $


    Suppose $ x \in f^{-1}(A) \bigcap f^{-1}(B)$, by 
     definition, then 
    $ x \in f^{-1}(A) \land x \in f^{-1}(B)$, then it is
    the same as
    $ f(x) \in A \land f(x) \in B$, so 
    $ f(x) \in A \bigcap B $, in this case,
    $ x \in f^{-1}(A \bigcap B)$. So $R.H.S \subseteq L.H.S $

    \vspace*{2cm}
    \textbf{Based on the conclusions on (1) and (2)},
    $ f^{-1}(A \bigcap B) = f^{-1}
    (A) \bigcap f^{-1}(B)$.
\end{proof}

\newpage
\section*{Problem 5}

\begin{proof}
    Notice that $ x = x + y-y, y=y+x-x$, So
    \begin{align}
         \vert x + x \vert &= \vert x-y+x+y \vert \\
          2\vert x\vert & \leq \vert x-y \vert + \vert x+y \vert \hspace*{0.4cm} \textit{By triangle inequality}
    \end{align}

    We can do the same operation on $y$
    \begin{align}
        \vert y + y \vert &= \vert y-x+y+x \vert \\
         2\vert y\vert & \leq \vert y-x \vert + \vert y+x \vert \hspace*{0.4cm} \textit{By triangle inequality}
   \end{align}

   Then, by the properties of absolute values, we know that
   $ \vert x-y \vert = \vert y-x \vert $

   We can do (1)+(2), then we will get

   \[ 2\vert x \vert + 2\vert y \vert \leq
   2\vert x-y \vert + 2\vert x+y \vert \]

   By substracting 2 on both sides, we get

   \[ \vert x \vert + \vert y \vert \leq
   \vert x-y \vert + \vert x+y \vert ,\hspace*{0.3cm}
   \text{which is exactly what we need to prove.}\]


        


\end{proof}





\end{document}
